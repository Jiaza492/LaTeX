\documentclass[12pt]{article}
\parindent=.25in

\setlength{\oddsidemargin}{0pt}
\setlength{\textwidth}{440pt}
\setlength{\topmargin}{0in}

\usepackage{amssymb}
\usepackage{amsfonts}
\usepackage{amsmath}

\title{IEOR 4150 Homework 2}
\author{Mengqi Zong $<mz2326@columbia.edu>$}

\begin{document}

\maketitle

\setlength{\parindent}{0in}

\section*{Problem 1}
4. \\
a) \\
b)
\begin{eqnarray*}
  P(2 < X \le 4)
  &=& F(4) - F(2) \\
  &=& 1 - \frac {11}{12} \\
  &=& \frac {1}{12}
\end{eqnarray*}
c)
\begin{eqnarray*}
  P(X < 3)
  &=& F(3^-) \\
  &=& \frac {11}{12}
\end{eqnarray*}
d)
\begin{eqnarray*}
  P(X == 1)
  &=& F(1) - F(1^-) \\
  &=& \frac {2}{3} - \frac {1}{2} \\
  &=& \frac {1}{6}
\end{eqnarray*}

6. We first compute factor $\lambda$
\begin{eqnarray*}
  \int_{-\infty}^{+\infty} f(x) dx
  &=& \int_{-\infty}^{0} f(x) dx + \int_{0}^{+\infty} f(x) dx \\
  &=& \int_{-\infty}^{0} 0 dx + \int_{0}^{+\infty} \lambda e^{-x/100} dx \\
  &=& \int_{0}^{+\infty} \lambda e^{-x/100} dx \\
  &=& -100 \lambda e^{-x/100}|_0^{+\infty} \\
  &=& 0 + 100 \lambda = 1 \\
  \Longrightarrow \lambda = \frac {1}{100}
\end{eqnarray*}
Then we get
\begin{eqnarray*}
  P(50 \le X \le 150)
  &=& \int_{50}^{150} f(x) dx \\
  &=& \int_{50}^{150} \frac {1}{100} e^{-x/100} dx \\
  &=& -e^{-x/100}|_{50}^{150} \\
  &=& e^{-1/2}-e^{-3/2}
\end{eqnarray*}
and
\begin{eqnarray*}
  P(X < 100)
  &=& \int_{-\infty}^{100} f(x) dx \\
  &=& \int_{-\infty}^{0} f(x) dx + \int_{0}^{100} f(x) dx \\
  &=& \int_{-\infty}^{0} 0 dx + \int_{0}^{100} \frac {1}{100} e^{-x/100} dx \\
  &=& \int_{0}^{100} \frac {1}{100} e^{-x/100} dx \\
  &=& -e^{-x/100}|_0^{100} \\
  &=& -e^{-1} + 1 \\
  &=& 1 - \frac {1}{e}
\end{eqnarray*}

8. We first compute factor $c$
\begin{eqnarray*}
  \int_{-\infty}^{+\infty} f(x) dx
  &=& \int_{-\infty}^{0} f(x) dx + \int_{0}^{+\infty} f(x) dx \\
  &=& \int_{-\infty}^{0} 0 dx + \int_{0}^{+\infty} c e^{-2x} dx \\
  &=& \int_{0}^{+\infty} c e^{-2x} dx \\
  &=& - \frac {1}{2} c e^{-2x}|_0^{+\infty} \\
  &=& \frac {1}{2} c = 1 \\
  \Longrightarrow c = 2
\end{eqnarray*}
Then we get
\begin{eqnarray*}
  P(X > 2)
  &=& \int_{2}^{+\infty} f(x) dx \\
  &=& \int_{2}^{+\infty} 2 e^{-2x} dx \\
  &=& -e^{-2x}|_{2}^{+\infty} \\
  &=& 0 + e^{-4} \\
  &=& \frac {1}{e^4}
\end{eqnarray*}

9.
\begin{eqnarray*}
  P(1,1)
  &=& \frac {{3 \choose 2}2!}{{5 \choose 2}2!}
  = \frac {3}{10} \\
  P(1,2)
  &=& \frac {{3 \choose 2} 2! {2 \choose 1} 1!} {{5 \choose 3} 3!}
  = \frac {1}{5} \\
  P(1,3)
  &=& \frac {{3 \choose 2} 2! {2 \choose 2} 2!} {{5 \choose 4} 4!}
  = \frac {1}{10} \\
  P(2,1)
  &=& \frac {{3 \choose 2} 2! {2 \choose 1} 1!} {{5 \choose 3} 3!}
  = \frac {1}{5} \\
  P(2,2)
  &=& \frac {{3 \choose 2} 2! {2 \choose 2} 2!} {{5 \choose 4} 4!}
  = \frac {1}{10} \\
  P(3,1)
  &=& \frac {{3 \choose 2} 2! {2 \choose 2} 2!} {{5 \choose 4} 4!}
  = \frac {1}{10}
\end{eqnarray*}

10.
a)
\begin{eqnarray*}
  \int_{0}^{2} \int_{0}^{1} f(x,y) dx dy
  &=& \int_{0}^{2} \int_{0}^{1} \frac {6}{7} (x^2 + \frac {xy}{2}) dx dy \\
  &=& \int_{0}^{2} \frac {6}{7} 
  (\frac{1}{3} x^3 + \frac {1}{4} x^2y)|_0^1 dy \\
  &=& \int_{0}^{2} \frac {6}{7} (\frac{1}{3} + \frac {1}{4} y) dy \\
  &=& \frac {6}{7} (\frac{1}{3} y + \frac {1}{8} y^2)|_0^2 \\
  &=& 1
\end{eqnarray*}
b)
\begin{eqnarray*}
  f_X(x)
  &=& \int_{0}^{2} f(x,y) dy \\
  &=& \int_{0}^{2} \frac {6}{7} (x^2 + \frac {xy}{2}) dy \\
  &=& (\frac {3}{7} x^2y + \frac {3}{14} xy^2)|_0^2 \\
  &=& \frac {6}{7} x^2 + \frac {6}{7} x
\end{eqnarray*}
c)
\begin{eqnarray*}
P(X < Y)
&=& \int_{0}^{1} \int_{0}^{y} f(x,y) dx dy 
+ \int_{1}^{2} \int_{0}^{1} f(x,y) dx dy \\
&=& \int_{0}^{1} \int_{0}^{y} \frac {6}{7} (x^2 + \frac {xy}{2}) dx dy
+  \int_{1}^{2} \int_{0}^{1} \frac {6}{7} (x^2 + \frac {xy}{2}) dx dy \\
&=& \int_{0}^{1} \frac {6}{7} (\frac{1}{3} x^3 + \frac {1}{4} x^2y)|_0^y dy
+ \int_{1}^{2} \frac{6}{7} (\frac{1}{3} x^3 + \frac {1}{4} x^2y)|_0^1 dy \\
&=& \int_{0}^{2} \frac{1}{2} y^3 dy + 
+ \int_{1}^{2} (\frac{2}{7} + \frac {3}{14} y) dy \\
&=& \frac{1}{8} y^4|_0^1 + (\frac{2}{7} y + \frac {3}{28} y^2)|_1^2 \\
&=& \frac {5}{8}
\end{eqnarray*}

12. \\
a) \\
When $x > 0$
\begin{eqnarray*}
  f_X(x)
  &=& \int_{0}^{+\infty} f(x,y) dy \\
  &=& \int_{0}^{+\infty} xe^{-(x+y)} dy \\
  &=& \int_{0}^{+\infty} xe^{-x}e^{-y} dy \\
  &=& xe^{-x} \cdot (-e^y|_0^{\infty}) \\
  &=& xe^{-x}
\end{eqnarray*}
When $x \le 0$
\begin{eqnarray*}
  f_X(x)
  &=& \int_{-\infty}^{0} f(x,y) dy \\
  &=& \int_{0}^{+\infty} 0 dy \\
  &=& 0
\end{eqnarray*}
So we get
\begin{equation*}
  f_X(x) =
  \begin{cases}
    xe^{-x} & x > 0,\\
    0 & \text{otherwise}
  \end{cases}
\end{equation*}
b) \\
When $y > 0$
\begin{eqnarray*}
  f_Y(y)
  &=& \int_{0}^{+\infty} f(x,y) dx \\
  &=& \int_{0}^{+\infty} xe^{-(x+y)} dx \\
  &=& \int_{0}^{+\infty} xe^{-x}e^{-y} dx \\
  &=& e^{-y} \cdot (-(xe^{-x}+e^{-x})|_0^{\infty}) \\
  &=& e^{-y}
\end{eqnarray*}
When $y \le 0$
\begin{eqnarray*}
  f_Y(y)
  &=& \int_{-\infty}^{0} f(x,y) dx \\
  &=& \int_{0}^{+\infty} 0 dx \\
  &=& 0
\end{eqnarray*}
So we get
\begin{equation*}
  f_Y(y) =
  \begin{cases}
    e^{-y} & y > 0,\\
    0 & \text{otherwise}
  \end{cases}
\end{equation*}
c) \\
\begin{eqnarray*}
  f_X(x)f_Y(y)
  &=& xe^{-x} \cdot e^{-y} \\
  &=& xe^{-(x+y)} \\
  &=& f(x,y)
\end{eqnarray*}
So X and Y are independent. \\

\section*{Problem 2}
{\bf Expectations} \\

a) Let $p$ denote the probability of head appears when tossing the coin. For this problem, we have $p = \frac {1}{2}$. \\
\begin{eqnarray*}
  P(X = n)
  &=& (1-p)^{n-1}p \\
  &=& \frac {1}{2^n}
\end{eqnarray*}
Then we get
\begin{eqnarray*}
  E[(2^X)]
  &=& \lim_{n \rightarrow +\infty} \sum_{x=1}^{n} P(x) \cdot 2^x \\
  &=& \lim_{n \rightarrow +\infty} \sum_{x=1}^{n} \frac {1}{2^x} \cdot 2^x \\
  &=& \lim_{n \rightarrow +\infty} \sum_{x=1}^{n} 1 \\
  &=& \lim_{n \rightarrow +\infty} \frac {(1+n)n}{2} \\
  &=& \infty
\end{eqnarray*}
b)
\begin{eqnarray*}
  Eu[(2^X)]
  &=& \lim_{n \rightarrow +\infty} \sum_{x=1}^{n} P(x) \cdot \ln(2^x) \\
  &=& \lim_{n \rightarrow +\infty} \sum_{x=1}^{n} \frac{x \ln 2}{2^x}\\
  &=& \lim_{n \rightarrow +\infty} \ln 2 \cdot
  (\frac {1 - \frac{1}{2^n}}{1 - \frac{1}{2}} - \frac{n}{2^n}) \\
  &=& \lim_{n \rightarrow +\infty} \ln 2 \cdot (2 - \frac {2+n}{2^n}) \\
  &=& 2\ln 2
\end{eqnarray*}
c)
\begin{eqnarray*}
  && u(w_c) = Eu[(2^X)] \\
  && \Rightarrow \ln (w_c) = 2 \ln 2 \\
  && \Rightarrow w_c = e^{2 \ln 2} = 2e^2
\end{eqnarray*}

 

\end{document}
