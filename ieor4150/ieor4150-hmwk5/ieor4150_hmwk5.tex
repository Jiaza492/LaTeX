\documentclass[12pt]{article}
\parindent=.25in

\setlength{\oddsidemargin}{0pt}
\setlength{\textwidth}{440pt}
\setlength{\topmargin}{0in}

\usepackage[dvips]{graphicx}
\usepackage{amssymb}
\usepackage{amsfonts}
\usepackage{amsmath}
\usepackage{verbatim}

\title{IEOR 4150 Homework 5}
\author{Mengqi Zong $<mz2326@columbia.edu>$}

\begin{document}

\maketitle

\setlength{\parindent}{0in}

\section*{Chapter 5}

43. \\

a)
\begin{eqnarray*}
  P \{ X \ge 6 \} &=& 0.4232 \\
  \Rightarrow P \{ X \le 6 \} &=& 1 - 0.4232 = 0.5678
\end{eqnarray*}

b)
\begin{eqnarray*}
  P \{ X \ge 3 \} &=& 0.8088 \\
  P \{ X \ge 9 \} &=& 0.1735 \\
  \Rightarrow   P \{ 3 \le X \le 9 \}
  &=& P \{ X \le 9 \} - P \{ X \le 3 \} \\
  &=& (1 - P \{ X \ge 9 \}) - (1 - P \{ X \ge 3 \}) \\
  &=& (1 - 0.1735) - (1 - 0.8088) \\
  &=& 0.6353
\end{eqnarray*}

46.
\begin{eqnarray*}
  P \{ T \ge 1 \} &=& 0.1733 \\
  P \{ T \le 2 \} &=& 1 - P \{ T \ge 2 \} = 1 - 0.04025 = 0.95975 \\
  P \{ -1 < T < 1 \} &=& 1 - 0.3466 = 0.6534
\end{eqnarray*}

47.
\begin{eqnarray*}
  T_n &=& \frac {Z}{\sqrt{\chi_n^2 / n}} \\
  \Rightarrow T_n^2 &=& \frac {Z^2}{\chi_n^2 / n} \\
  &=& \frac {Z^2 / 1}{\chi_n^2 / n} \\
  &=& F_{1,n}
\end{eqnarray*}

\section*{Chapter 6}

4. \\
a) For 34 bets, you will win if at least one of the 34 bets win. So
\begin{eqnarray*}
  P_a = 1 - (\frac{37}{38})^{34} = 0.5962
\end{eqnarray*}

b) Let $X_i$ denote the $i$th bet's outcome. Then we have
\begin{eqnarray*}
  E[X_i] &=& 1 \times \frac{1}{38} \times 35 + (-1) \times \frac{37}{38}
  = - \frac{1}{19} \\
  Var(X_i)
  &=& \frac{1 \times (35 - (- \frac{1}{19}))^2
    + 37 \times (-1 - (- \frac{1}{19}))^2} {38}
  = 33.2087
\end{eqnarray*}

Since 1000 is very large, we can assume the distribution of the final outcome is approximately a normal distribution with
\begin{eqnarray*}
  \mu &=& 1000 \times (- \frac{1}{19}) = 52.6316 \\
  \sigma^2 &=& 1000 \times 33.2087 = 33208.7
\end{eqnarray*}

Then we get
\begin{eqnarray*}
  P_b
  &=& \Phi \left( \frac{0 - \mu}{\sigma} \right) \\
  &=& \Phi \left( \frac{0 - 52.6316}{\sqrt {33208.7}} \right) \\
  &=& \Phi \left( -0.2888 \right) \\
  &=& 1 - 0.6141 \\
  &=& 0.3859
\end{eqnarray*}

c) \\
Similar to b), we can assume the distribution of the final outcome is approximately a normal distribution with
\begin{eqnarray*}
  \mu &=& 100000 \times (- \frac{1}{19}) = 5263.16 \\
  \sigma^2 &=& 100000 \times 33.2087 = 3320870
\end{eqnarray*}

Then we get
\begin{eqnarray*}
  P_b
  &=& \Phi \left( \frac{0 - \mu}{\sigma} \right) \\
  &=& \Phi \left( \frac{0 - 5263.16}{\sqrt {3320870}} \right) \\
  &=& \Phi \left( -2.8882 \right) \\
  &=& 1 - 0.9981 \\
  &=& 0.0019
\end{eqnarray*}

7. \\

Let $X_i$ denote the $i$th bet's outcome. Then we have
\begin{eqnarray*}
  E[X_i] &=& \frac{\sum_{i=1}^6 i} {6} = 3.5 \\
  Var(X_i) &=& \frac{\sum_{i=1}^6 (i - 3.5)}{6} = 2.9176
\end{eqnarray*}

Since 140 is very large, we can assume the distribution of the final outcome is approximately a normal distribution with
\begin{eqnarray*}
  \mu &=& 140 \times 3.5 = 490 \\
  \sigma^2 &=& 140 \times 2.9176 = 408.464
\end{eqnarray*}

Then we get
\begin{eqnarray*}
  P \{ require more than 140 rolls \}
  &=& P \{ the outcome of 140 rolls is less than 400 \} \\
  &=& \Phi(\frac{400 - \mu}{\sigma}) \\
  &=& \Phi(\frac{400 - 490}{\sqrt {408.464}}) \\
  &=& \Phi(-4.4531) \\
  &=& 1 - 0.999 \\
  &=& 0.001
\end{eqnarray*}

10.
\begin{eqnarray*}
  t &=& \frac {3.1 - 2.2}{0.3} = 3
\end{eqnarray*}

Then we get
\begin{eqnarray*}
  P\{ X > 3.1 \} = 0.0017
\end{eqnarray*}

11. \\
\begin{eqnarray*}
  t &=& \frac {525 - 500}{80} = 0.3125
\end{eqnarray*}

a)
\begin{eqnarray*}
  P\{ X > 525 \} = 0.38515
\end{eqnarray*}

b)
\begin{eqnarray*}
  P\{ X > 525 \} = 0.37935
\end{eqnarray*}

c)
\begin{eqnarray*}
  P\{ X > 525 \} = 0.37825
\end{eqnarray*}

d)
\begin{eqnarray*}
  P\{ X > 525 \} = 0.37785
\end{eqnarray*}

16. \\

A Poisson random variable with parameter $\lambda$ can be thought of as a sum of $\lambda$ independent Poisson random variables, each with parameter one. Based on the central limit theorem, when $\lambda$ is very large, the random variable can be well approximated by a normal random variable with the same mean and variance.

\begin{eqnarray*}
  P_{exact}
  &=& \sum_{i=0}^{k} e^{-\lambda} \frac{\lambda^i}{i!} \\
  &=& \sum_{i=0}^{116} e^{-100} \frac{100^{116}}{116!} \\
  &=& 0.9478
\end{eqnarray*}

\begin{eqnarray*}
  P_{approx}
  &=& \Phi (\frac {116 - 100}{100}) \\
  &=& \Phi (0.16) \\
  &=& 0.5636
\end{eqnarray*}

18. \\
a) Since the temperature is normally distributed, we know that $(n-1) S^2 / \sigma^2$ having a chi-square distribution with $n-1$ degrees of freedom. So we get
\begin{eqnarray*}
  P \{ S^2 / \sigma^2 \le 1.8 \}
  &=& P \{ (5-1) S^2 / \sigma^2 \le (5-1) \times 1.8 \} \\
  &=& 1 - P \{ 4 S^2 / \sigma^2 \ge 7.2 \} \\
  &=& 1 - 0.1257 \\
  &=& 0.8743
\end{eqnarray*}

b) 
\begin{eqnarray*}
  P \{ 0.85 \le S^2 / \sigma^2 \le 1.15 \}
  &=& P \{ 3.4 \le 4 S^2 / \sigma^2 \le 4.6 \} \\
  &=& P \{ 4 S^2 / \sigma^2 \le 4.6 \} - P \{ 4 S^2 / \sigma^2 \le 3.4 \} \\
  &=& (1 - P \{ 4 S^2 / \sigma^2 \ge 4.6 \})
  - (1 - P \{ 4 S^2 / \sigma^2 \ge 3.4 \}) \\
  &=& P \{ 4 S^2 / \sigma^2 \ge 3.4 \} - P \{ 4 S^2 / \sigma^2 \ge 4.6 \} \\
  &=& 0.4932 - 0.3309 \\
  &=& 0.1632
\end{eqnarray*}

20. \\
we know that for sample that is normally distributed, then $(n-1) S^2 / \sigma^2$ have a chi-square distribution with $n-1$ degrees of freedom. For the first sample, we get $9 S_1^2 / 4$ have a chi-square distribution with $9$ degrees of freedom. For the second sample, we get $2 S_2^2$ have a chi-square distribution with $4$ degrees of freedom. Then we get
\begin{eqnarray*}
  F_{9,4}
  &=& \frac {9 S_1^2 / 4 / 9}{2 S_2^2 / 4} \\
  &=& \frac {S_1^2}{2 S_2^2}
\end{eqnarray*}

have a F-distribution with 9 and 4 degrees. \\

We then get
\begin{eqnarray*}
  P \{ \text{exceeds} \}
  &=& P \{ S_1^2 / 2 S_2^2 < 1/2 \} \\
  &=& 1 - P \{ F_{9,4} > 1/2 \} \\
  &=& 1 - 0.8218 \\
  &=& 0.1782
\end{eqnarray*}

\end{document}
