\documentclass[12pt]{article}
\parindent=.25in

\setlength{\oddsidemargin}{0pt}
\setlength{\textwidth}{440pt}
\setlength{\topmargin}{0in}

\usepackage[dvips]{graphicx}
\usepackage{amssymb}
\usepackage{amsfonts}
\usepackage{amsmath}

\title{IEOR 4150 Homework 3}
\author{Mengqi Zong $<mz2326@columbia.edu>$}

\begin{document}

\maketitle

\setlength{\parindent}{0in}

\section*{Chapter 4}

22. \\
From the description, we can get the sample space is
\begin{eqnarray*}
  S = \{ HHH, HHT, HTH, HTT, THH, THT, TTH, TTT\}
\end{eqnarray*}
where $H$ means head and $T$ means tail. \\

We first compute the probabilities associated with the values that X can take on
\begin{eqnarray*}
  P(X = -3) &=& \frac {|\{ TTT \}|}{|S|} = \frac {1}{8} \\
  P(X = -1) &=& \frac {|\{ HTT, THT, TTH \}|}{|S|} = \frac {3}{8} \\
  P(X =  1) &=& \frac {|\{ HHT, HTH, THH \}|}{|S|} = \frac {3}{8} \\
  P(X =  3) &=& \frac {|\{ HHH \}|}{|S|} = \frac {1}{8} \\
\end{eqnarray*}
Then we get
\begin{eqnarray*}
  E(X) &=& \sum_x x Pr(x) \\
       &=& (-3) \cdot \frac {1}{8} + (-1) \cdot \frac {3}{8}
           + 1 \cdot \frac {3}{8} + 3 \cdot \frac {1}{8} \\
       &=& 0
\end{eqnarray*}

34. \\
a)
\begin{eqnarray*}
  F(x) &=& \int_0^x e^{-x} dx \\
       &=& -e^{-x}|_0^x \\
       &=& 1 - e^{-x}
\end{eqnarray*}
To find the median, we have
\begin{eqnarray*}
  F(m) &=& \frac {1}{2} \\
  1 - e^{-m} &=& \frac {1}{2} \\ 
  e^{-m} &=& \frac {1}{2} \\
  m &=& \ln 2
\end{eqnarray*}
b)
\begin{eqnarray*}
  F(x) &=& \int_0^x 1 dx \\
       &=& x|_0^x \\
       &=& x
\end{eqnarray*}
To find the median, we have
\begin{eqnarray*}
  F(m) &=& \frac {1}{2} \\
  x &=& \frac {1}{2} \\ 
\end{eqnarray*}

45. \\
a) The marginal probability distribution of $X_1$ is
\begin{eqnarray*}
  f(X_1 = 0) &=& f(X_1 = 0, X_2 = 1) + f(X_1 = 0, X_2 = 2) = \frac {3}{16} \\ 
  f(X_1 = 1) &=& f(X_1 = 1, X_2 = 1) + f(X_1 = 1, X_2 = 2) = \frac {1}{8} \\ 
  f(X_1 = 2) &=& f(X_1 = 2, X_2 = 1) + f(X_1 = 2, X_2 = 2) = \frac {1}{4} \\ 
  f(X_1 = 3) &=& f(X_1 = 3, X_2 = 1) + f(X_1 = 3, X_2 = 2) = \frac {3}{8}
\end{eqnarray*}
The marginal probability distribution of $X_2$ is
\begin{eqnarray*}
  f(X_2 = 1) &=& f(X_1 = 0, X_2 = 1) + f(X_1 = 1, X_2 = 1) 
    + f(X_1 = 2, X_2 = 1) + f(X_1 = 3, X_2 = 1) = \frac {1}{2} \\ 
  f(X_2 = 2) &=& f(X_1 = 0, X_2 = 2) + f(X_1 = 1, X_2 = 2)
    + f(X_1 = 2, X_2 = 2) + f(X_1 = 3, X_2 = 2) = \frac {1}{2}
\end{eqnarray*}
b)
\begin{eqnarray*}
  E[(X_1)] &=& \sum_{x_1} x_1 f(x_1) \\
           &=& 0 \cdot \frac {3}{16} + 1 \cdot \frac{1}{8}
               + 2 \cdot \frac {1}{4} + 3 \cdot \frac{3}{8} \\
           &=& 1.75 \\
  E[(X_2)] &=& \sum_{x_2} x_2 f(x_2) \\
           &=& 1 \cdot \frac{1}{2} + 2 \cdot \frac {1}{2} \\
           &=& 1.5 \\
  Var(X_1) &=& E[(X_1 - \mu_{X_1})^2] \\
           &=& (0 - 1.75)^2 \cdot \frac {3}{16}
               + (1 - 1.75)^2 \cdot \frac {1}{8}
               + (2 - 1.75)^2 \cdot \frac {1}{4}
               + (3 - 1.75)^2 \cdot \frac {3}{8} \\
           &=& \frac {319}{196} \\
  Var(X_2) &=& E[(X_1 - \mu_{X_2})^2] \\
           &=& (1 - 1.5)^2 \cdot \frac {1}{2}
               + (2 - 1.5)^2 \cdot \frac {1}{2} \\
           &=& \frac {1}{4} \\
  Cov(X_1, X_2)
  &=& E[(X_1 - \mu_{X_1})(X_2 - \mu_{X_2})] \\
  &=& (0 - 1.75)(1 - 1.5) \cdot \frac {3}{16} \cdot \frac {1}{2}
      + (1 - 1.75)(1 - 1.5) \cdot \frac {1}{8} \cdot \frac {1}{2} \\
  &+& (2 - 1.75)(1 - 1.5) \cdot \frac {1}{4} \cdot \frac {1}{2}
      + (3 - 1.75)(1 - 1.5) \cdot \frac {3}{8} \cdot \frac {1}{2} \\
  &+& (0 - 1.75)(2 - 1.5) \cdot \frac {3}{16} \cdot \frac {1}{2}
      + (1 - 1.75)(2 - 1.5) \cdot \frac {1}{8} \cdot \frac {1}{2} \\
  &+& (2 - 1.75)(2 - 1.5) \cdot \frac {1}{4} \cdot \frac {1}{2}
      + (3 - 1.75)(2 - 1.5) \cdot \frac {3}{8} \cdot \frac {1}{2} \\
  &=& 0
\end{eqnarray*}

46. \\
a)
\begin{eqnarray*}
  \int_{-\infty}^{+\infty} f(z) dz
  &=& \int_{0}^{1} kz^2 dz + \int_{1}^{\frac {4}{3}} 1 dz \\
  &=& \frac {k}{3} z^3|_0^1 + z|_1^{\frac {4}{3}} \\
  &=& \frac {k}{3} (1 - 0) + (\frac {4}{3} - 1) \\
  &=& \frac {k}{3} + \frac{1}{3} \\
  &=& \frac {k+1}{3} = 1 \\
  &\Longrightarrow& k = 2
\end{eqnarray*}
b)
\begin{eqnarray*}
  P(b)
  &=&\int_{0}^{1} 2 z^2 dz \\
  &=& \frac {2}{3} z^3|_0^1 \\
  &=& \frac {2}{3}
\end{eqnarray*}
c)
\begin{eqnarray*}
  E[(z)]
  &=& \int_{-\infty}^{+\infty} zf(z) dz \\
  &=& \int_{0}^{1} 2z^3 dz + \int_{1}^{\frac {4}{3}} z dz \\
  &=& \frac {1}{2} z^4|_0^1 + \frac {1}{2} z^2|_1^{\frac {4}{3}} \\
  &=& \frac {1}{2} + \frac {7}{18} \\
  &=& \frac {8}{9} \\
  Var(z)
  &=& \int_{-\infty}^{+\infty} (z - \frac {8}{9})^2 f(z) dz \\
  &=& \int_{0}^{1} (z - \frac {8}{9})^2 2z^2 dz
  + \int_{1}^{\frac {4}{3}} (z - \frac {8}{9})^2 dz \\
  &=& \int_{0}^{1} (2 z^4 - \frac {32}{9} z^3 + \frac {128}{81} z^2) dz
  + \int_{1}^{\frac {4}{3}} (z^2 - \frac {16}{9} z + \frac {64}{81}) dz \\
  &=& (\frac {2}{5} z^5 - \frac {8}{9} z^4 + \frac {128}{243} z^3)|_0^1
  + (\frac{1}{3} z^3 - \frac{8}{9} z^2 + \frac{64}{81} z)|_1^{\frac {4}{3}} \\
  &=& \frac {2}{5} - \frac {8}{9} + \frac {128}{243}
  + \frac {1}{3} (\frac{64}{27} - 1) - \frac{8}{9}(\frac{16}{9} - 1)
  + \frac{64}{81} (\frac {4}{3} - 1) \\
  &=& \frac {1}{15}
\end{eqnarray*}

50. \\



53. \\

55. \\


\end{document}
