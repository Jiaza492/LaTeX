\documentclass[12pt]{article}

\parindent=.25in
\setlength{\oddsidemargin}{0pt}
\setlength{\textwidth}{440pt}
\setlength{\topmargin}{0in}

\usepackage{amsmath}
\usepackage[dvips]{graphicx}
\usepackage{verbatim}
\usepackage{appendix}

\title{COMS 4236 Homework 1}
\author{Mengqi Zong $<mz2326@columbia.edu>$}

\begin{document}

\maketitle

\setlength{\parindent}{0in}

\section*{Problem 1}


a. The transition table is shown in Table-\ref{tab:p1_a}. And here is
the brief explanation:

\begin{enumerate}
\item If the last bit is 0 ($s_0$): change the bit to 1 ($s_1$),
  scan the next input.
\item If the last bit is 1 ($s_1$), change the bit to 0, and
  worktape moves back 1 bit ($p_1$).
  \begin{itemize}
  \item If the bit is 0 ($p_1$), change it to 1 ($p_0$). Worktape
    moves to the last bit.
  \item If the bit is 1 ($p_1$), change it to 0. Worktape moves back
    1 bit ($p_1$).
  \item If the bit is $\triangleright$, work tape move to the first
    bit ($p_f$) and flip to 1 ($p'_f$). Then go to the first $\sqcup$
    bit and set it to 0.
  \end{itemize}
\end{enumerate}

\begin{table}[ht!]
\begin{center}
\begin{tabular}{|ccc|c|}
  \hline

  $p \in K$ & $\sigma_1 \in \Sigma$ & $\sigma_2 \in \Sigma$
  & $\delta(p1,\sigma_1,\sigma_2)$ \\
  \hline
  $s_0$  &    1     &    0     &  $(s_1, 1, \to, 1, -)$  \\
  $s_0$  & $\sqcup$ &    0     &  $(h, \sqcup, -, 0, -)$  \\
  $s_1$  &    1     &    1     &  $(p_1, 1, -, 0, \gets)$  \\
  $s_1$  & $\sqcup$ &    1     &  $(h, \sqcup, -, 1, -)$  \\
  $s_1$  &    1     &    0     &  $(p_0, 1, -, 1, \to)$  \\
  $p_0$  &    1     &    0     &  $(p_0, 1, -, 0, \to)$ \\
  $p_0$  &    1     &    1     &  $(p_0, 1, -, 1, \to)$ \\
  $p_0$  &    1     & $\sqcup$ &  $(p'_0, 1, -, \sqcup, \gets)$ \\
  $p'_0$ &    1     &    0     &  $(s_0, 1, \to, 0, -)$ \\
  $p'_0$ &    1     &    1     &  $(s_1, 1, \to, 1, -)$ \\
  $p_1$  &    1     &    0     &  $(p_0, 1, -, 1, \to)$ \\
  $p_1$  &    1     &    1     &  $(p_1, 1, -, 0, \gets)$ \\
  $p_1$  &    1     & $\triangleright$ &  $(p_f, 1, -, \sqcup, \to)$ \\
  $p_f$  &    1     &    0     &  $(p'_f, 1, -, 1, \to)$ \\
  $p'_f$ &    1     &    0     &  $(p'_f, 1, -, 0, \to)$ \\
  $p'_f$ &    1     &    1     &  $(p'_f, 1, -, 1, \to)$ \\
  $p'_f$ &    1     & $\sqcup$ &  $(s_0, 1, \to, 0, -)$ \\

  \hline
\end{tabular}
\end{center}
\caption{Problem 1-a: 2-string Turing machine for length
  counting \label{tab:p1_a}}  
\end{table}

b. Using amortized analysis, we can show that given any input string
with length $n$, its $k^{th}$ $(0 \le m \le \log n)$ digit will be
flipped  $ 1 / 2^k \cdot n$ times. And the total running time for the
string is

\begin{eqnarray}
  T(n) &=&    \sum_{k = 0}^{\log n} {\frac {1}{2^k} \cdot n \cdot k}
              \nonumber \\
       &=&  n \sum_{k = 0}^{\log n} {\frac {1}{2^k} \cdot k} 
              \nonumber \\
       &\leq& n (1 + \sum_{k = 0}^{\infty} {\frac {1}{2^k} \cdot k}) 
              \nonumber \\
       &\leq& n (1 + 2) \nonumber \\
       &=&    O(n)
\end{eqnarray}

So the time complexity of the TM is $O(n)$.

\section*{Problem 2}

a. Let the TM have two heads, $h_1$ and $h_2$. First, move head
$h_2$ to the last symbol of the input. Second, begin to compare the
symbol on the two heads: If they are the same, then keep going, until
both of them reach the end ($h_1$ reaches the right-end and $h_2$
reaches the left-end); If they are not the same, half and output
``No''. If both heads reaches the end, then half and ouput ``Yes''. In
this case, it takes $O(n)$ to run the algorithm.\\

b. Suppose a multihead Turing machine has $l$ heads. Since it is
possible for a tape to have multiple heads, then we need some extra
space to store the position of each head.\\

For every work tape, since a work tape can at most have S(n) space,
then it takes $O(\log {S(n)})$ to store each head's position. For the
input tape, since it takes $n$ space, then it takes $O(\log {n})$ to
store each head's position. As a result, it takes $O(\log {S(n)} +
\log {n})$ space to store the heads' positions.\\

In total, the multihead Turing machine takes $O(S(n) + \log {S(n)} +
\log {n})$ space. Simplify a little, we get $O(S(n) + \log {n})$.\\

About time complexity, let $S^{'}(n) = O(S(n) + \log {n})$. Since
the number of configurations of a TM M with space $S(n)$ is at most
$n \cdot c^{S(n)}$ for some constant c that depends on M, replace
$S(n)$ with $S^{'}(n)$ we get the time complexity of a multihead
Turing machine is $O(n \cdot c^{S(n) + \log {n}})$.

\section*{Problem 3}

a. To solve this problem, I use a i-o Turing machine with 7 work
tapes. Here are the 7 work tapes:

\begin{itemize}
\item $t_1$: store input $a$.
\item $t_2$: store input $b$.
\item $t_3$: store $length(a)$.
\item $t_4$: store $length(b)$.
\item $t_5$: store the counter (will explain later).
\item $t_6$: store carry flag (Only 1 bit, 0 means no carry and 1
  means there is a carry).
\item $t_7$: store current result with leading zeros.
\end{itemize}

Here is the algorithm:

\begin{enumerate}
\item Count the length of a and b, store them respectively on $t_3$ and
  $t_4$. Also check the input format, if the format is wrong, then half
  and print \# on the output. 
\item Store $a$ on $t_1$ and store $b$ on $t_2$. If $length(a) \neq
  length(b)$, then put leading zeros to the respective tape which
  stores the input with shorter length.
\item Initialize $t_5$ the counter to be $max(length(a), length(b))$. 
\item Initialize $t_6$ the carry flag to be $0$.
\item Since the output can take at most $max(leng(a), length(b)) + 1$
  space, initialize $t_7$ with $max(leng(a), length(b)) + 1$ zeros.
\item With each head of $t_1$, $t_2$ and $t_7$ point at their last bit
  (the least significant bit), now let's start the addition: according
  to the different input of current bit of $a$, $b$ and the carry flag
  $t_6$, set the bit of $t_7$ and the also the carry flag $t_6$. Then
  counter $t_5$ decrement by 1, and the heads of $t_1$, $t_2$, and
  $t_7$ move backwards by one bit.
\item Repeat the former step until counter $t_5$ reaches 0. And if the
  carry flag is 1, then $t_7$ move backwards by 1 bit, and set the
  current bit to 1.
\item Print the output from $t_7$ to the output tape. Depends on
  different requirements, it is possible to eliminate the leading
  zeros in the output. Note that we have to consider the condition
  when output is 0, but that's not a problem.
\end{enumerate}

b. Compare with a, we only need 6 tapes at this time. And instead of
storing the exact input $a$ and $b$, we just need to store the
position of the current digit of input $a$ and $b$. Here are the 6
tapes:

\begin{itemize}
\item $t_1$: store the current digit position of input $a$.
\item $t_2$: store the current digit position of input $b$.
\item $t_3$: store $length(a)$.
\item $t_4$: store $length(b)$.
\item $t_5$: store the counter.
\item $t_6$: store carry flag (1 bit). And 0 means no carry, 1 means
  there is a carry.
\end{itemize}

Here is the algorithm:

\begin{enumerate}
\item Count the length of a and b, store them respectively on $t_3$ and
  $t_4$. Also check the input format, if the format is wrong, then half
  and print \# on the output. 
\item Store the position of the least significant bit of $a$ on $t_1$
  and store the position of the least significant bit of $b$ on
  $t_2$.
\item Initialize $t_5$ the counter to be $max(length(a), length(b))$. 
\item Initialize $t_6$ the carry flag to be $0$.
\item Now let's start the addition from the least significant bit:
  according to the different input of current bit of $a$, $b$ and the
  carry flag $t_6$, set the bit of the output and the also the carry
  flag $t_6$. Then counter $t_5$ decrement by 1, length $t_3$ and $t4$
  decrement by 1, and so does the head position $t_1$ and $t_2$. If
  one of $t_3$ and $t_4$ reaches 0, then the output is only depends on
  the not-yet-end input and the carry flag. Also, each step the head
  of the output move forward by 1 bit.
\item Repeat the former step until counter $t_5$ reaches 0. And if the
  carry flag is 1, then output move forward 1 bit, and set the current
  bit to 1.
\end{enumerate}

Since every work tape takes at most $O(\log n)$ space, this Turing
machine has a space complexity $O(\log n)$.\\

c. Compared with part b, we still use a 6-tape Turing machine. Here
are the 6 tapes.

\begin{itemize}
\item $t_1$: store the current digit position of input $a$.
\item $t_2$: store the current digit position of input $b$.
\item $t_3$: store $length(a)$.
\item $t_4$: store $length(b)$.
\item $t_5$: store the counter.
\item $t_6$: store carry flag (1 bit). And 0 means no carry, 1 means
  there is a carry.
\end{itemize}

As to the algorithm, the key of calculating the sum from the most
significant bit to the least significant bit is to get the correct
carry flag. Here is the algorithm:

\begin{enumerate}
\item Count the length of a and b, store them respectively on $t_3$ and
  $t_4$. Also check the input format, if the format is wrong, then half
  and print \# on the output. 
\item Store the position of the most significant bit of $a$ on $t_1$
  and store the position of the most significant bit of $b$ on
  $t_2$.
\item Initialize $t_5$ the counter to be $max(length(a),
  length(b)) + 1$. Here, the counter means the bit the addition is
  currently dealing with. 
\item Initialize $t_6$ the carry flag to be $0$.
\item Now let's start the addition from the most significant bit:
  \begin{enumerate}
  \item Since counter $t_5$ is the bit the addition that is now
    dealing with, then we need to check $t_3$ and $t_4$ to see if
    input a or b has such a bit. If $t_3$ or $t_4$ is less than $t_5$,
    it means the input does not have such a bit, or its ${t_5}^{th}$
    bit is $0$.
  \item With two input's current bits, we still need to know the carry
    flag. Here is the basic idea of how to compute the carry flag:
    
    Look at the bits one step away from the current bit of both inputs
    (If they don't have such bit, then this bit count as 0). If both
    bits are 1, then carry flag is 1. If both bits are 0, then carry
    flag is 0. But if one bit is 1 and other bit is 0, then we need to
    continue looking, that is, repeat this procedure whether both bits
    are 0 or 1, or reach the least significant bit of both input. If
    we reach both inputs' least significant bit and still 1 bit is 0
    and the other bit 1, then the carry flag is 0.
  \item With the two current bits of $a$ and $b$, also the carry flag,
    we can compute the current bit of the output. Note that we don't
    care about whether this will cause a carry or not since we
    calculate carry flag backwards. And at the first run, $t_5 =
    max(length(a), length(b)) + 1$, and both inputs' current bits is
    0. So the first bit of the output only depends on the carry flag
    we compute. 
  \end{enumerate}
\item Repeat the former step until counter $t_5$ reaches 0.
\end{enumerate}

About the time complexity, since both cases b and c have space
complexity $O(\log n)$, so the time complexity of both cases is
$O(c^{\log n})$, which is polynomial.

\section*{Problem 4}

a.

\end{document}
