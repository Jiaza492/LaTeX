\documentclass[12pt]{article}

\usepackage{amsmath}
\usepackage[dvips]{graphicx}
\usepackage{verbatim}
\usepackage{appendix}

\title{COMS 4236 Homework 1}
\author{Mengqi Zong $<mz2326@columbia.edu>$}

\begin{document}

\maketitle

\setlength{\parindent}{0in}

\section{Problem 1}


a. The transition table is shown in Table-\ref{tab:p1_a}. And here is
the brief explanation:

\begin{enumerate}
\item If the last bit is 0 ($s_0$): change the bit to 1 ($s_1$),
  scan the next input.
\item If the last bit is 1 ($s_1$), change the bit to 0, and
  worktape moves back 1 bit ($p_1$).
  \begin{itemize}
  \item If the bit is 0 ($p_1$), change it to 1 ($p_0$). Worktape
    moves to the last bit.
  \item If the bit is 1 ($p_1$), change it to 0. Worktape moves back
    1 bit ($p_1$).
  \item If the bit is $\triangleright$, work tape move to the first
    bit ($p_f$) and flip to 1 ($p'_f$). Then go to the first $\sqcup$
    bit and set it to 0.
  \end{itemize}
\end{enumerate}

\begin{table}[ht!]
\begin{center}
\begin{tabular}{|ccc|c|}
  \hline

  $p \in K$ & $\sigma_1 \in \Sigma$ & $\sigma_2 \in \Sigma$
  & $\delta(p1,\sigma_1,\sigma_2)$ \\
  \hline
  $s_0$  &    1     &    0     &  $(s_1, 1, \to, 1, -)$  \\
  $s_0$  & $\sqcup$ &    0     &  $(h, \sqcup, -, 0, -)$  \\
  $s_1$  &    1     &    1     &  $(p_1, 1, -, 0, \gets)$  \\
  $s_1$  & $\sqcup$ &    1     &  $(h, \sqcup, -, 1, -)$  \\
  $s_1$  &    1     &    0     &  $(p_0, 1, -, 1, \to)$  \\
  $p_0$  &    1     &    0     &  $(p_0, 1, -, 0, \to)$ \\
  $p_0$  &    1     &    1     &  $(p_0, 1, -, 1, \to)$ \\
  $p_0$  &    1     & $\sqcup$ &  $(p'_0, 1, -, \sqcup, \gets)$ \\
  $p'_0$ &    1     &    0     &  $(s_0, 1, \to, 0, -)$ \\
  $p'_0$ &    1     &    1     &  $(s_1, 1, \to, 1, -)$ \\
  $p_1$  &    1     &    0     &  $(p_0, 1, -, 1, \to)$ \\
  $p_1$  &    1     &    1     &  $(p_1, 1, -, 0, \gets)$ \\
  $p_1$  &    1     & $\triangleright$ &  $(p_f, 1, -, \sqcup, \to)$ \\
  $p_f$  &    1     &    0     &  $(p'_f, 1, -, 1, \to)$ \\
  $p'_f$ &    1     &    0     &  $(p'_f, 1, -, 0, \to)$ \\
  $p'_f$ &    1     &    1     &  $(p'_f, 1, -, 1, \to)$ \\
  $p'_f$ &    1     & $\sqcup$ &  $(s_0, 1, \to, 0, -)$ \\

  \hline
\end{tabular}
\end{center}
\caption{Problem 1-a: 2-string Turing machine for length
  counting \label{tab:p1_a}}  
\end{table}

b. Using amortized analysis, we can show that given any input string
with length $n$, its $k^{th}$ $(0 \le m \le \log n)$ digit will be
flipped  $ 1 / 2^k \cdot n$ times. And the total running time for the
string is

\begin{eqnarray}
  T(n) &=&    \sum_{k = 0}^{\log n} {\frac {1}{2^k} \cdot n \cdot k}
              \nonumber \\
       &=&  n \sum_{k = 0}^{\log n} {\frac {1}{2^k} \cdot k} 
              \nonumber \\
       &\leq& n (1 + \sum_{k = 0}^{\infty} {\frac {1}{2^k} \cdot k}) 
              \nonumber \\
       &\leq& n (1 + 2) \nonumber \\
       &=&    O(n)
\end{eqnarray}

So the time complexity of the TM is $O(n)$.

\section{Problem 2}
a. Let the TM have two heads, $h_1$ and $h_2$. First, move head
$h_2$ to the last symbol of the input. Second, begin to compare the
symbol on the two heads: If they are the same, then keep going, until
both of them reach the end ($h_1$ reaches the right-end and $h_2$
reaches the left-end); If they are not the same, half and output
``No''. If both heads reaches the end, then half and ouput ``Yes''. In
this case, it takes $O(n)$ to run the algorithm.\\

b. Suppose a multihead Turing machine has $l$ heads. Since it is
possible for a tape to have multiple heads, then we need some extra
space to store the position of each head.\\

For every work tape, since a work tape can at most have S(n) space,
then it takes $O(\log {S(n)})$ to store each head's position. For the
input tape, since it takes $n$ space, then it takes $O(\log {n})$ to
store each head's position. As a result, it takes $O(\log {S(n)} +
\log {n})$ space to store the heads' postions.\\

In total, the multihead Turing machine takes $O(S(n) + \log {S(n)} +
\log {n})$ space. Simplify a little, we get $O(S(n) + \log {n})$.\\

About time complexity, let $S^{'}(n) = O(S(n) + \log {n})$. Since
the number of configurations of a TM M with space $S(n)$ is at most
$n \cdot c^{S(n)}$ for some constant c that depends on M, replace
$S(n)$ with $S^{'}(n)$ we get the time complexity of a multihead
Turing machine is $O(n \cdot c^{S(n) + \log {n}})$.

\section{Problem 3}



\section{Problem 4}


\appendix
\appendixpage
\addappheadtotoc

\end{document}
