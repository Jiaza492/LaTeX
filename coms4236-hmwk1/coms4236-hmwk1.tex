\documentclass[12pt]{article}

\usepackage{amsmath}
\usepackage[dvips]{graphicx}
\usepackage{verbatim}
\usepackage{appendix}

\title{COMS 4236 Homework 1}
\author{Mengqi Zong $<mz2326@columbia.edu>$}

\begin{document}

\maketitle

\setlength{\parindent}{0in}

\section{Problem 1}


a.\\
The transition table is shown in Table-\ref{tab:p1_a}. And here is the
brief explanation:

\begin{enumerate}
\item If the last bit is 0 ($s_0$): change the bit into 1 ($s_1$),
  scan the next input.
\item If the last bit is 1 ($s_1$), change the bit into 0, and work
  tape moves back 1 bit ($p_1$).
  \begin{itemize}
  \item If the bit is 0 ($p_1$), change it into 1 ($p_0$). Work tape
    moves to the last bit.
  \item If the bit is 1 ($p_1$), change it into 0. Work tape moves back
    1 bit ($p_1$).
  \item If the bit is $\triangleright$, work tape move to the first
    bit ($p_f$) and flip to 1 ($p'_f$). Then go to the first $\sqcup$
    bit and set it to 0.
  \end{itemize}
\end{enumerate}

\begin{table}[ht!]
\begin{center}
\begin{tabular}{|ccc|c|}
  \hline

  $p \in K$ & $\sigma_1 \in \Sigma$ & $\sigma_2 \in \Sigma$
  & $\delta(p1,\sigma_1,\sigma_2)$ \\
  \hline
  $s_0$  &    1     &    0     &  $(s_1, 1, \to, 1, -)$  \\
  $s_0$  & $\sqcup$ &    0     &  $(h, \sqcup, -, 0, -)$  \\
  $s_1$  &    1     &    1     &  $(p_1, 1, -, 0, \gets)$  \\
  $s_1$  & $\sqcup$ &    1     &  $(h, \sqcup, -, 1, -)$  \\
  $s_1$  &    1     &    0     &  $(p_0, 1, -, 1, \to)$  \\
  $p_0$  &    1     &    0     &  $(p_0, 1, -, 0, \to)$ \\
  $p_0$  &    1     &    1     &  $(p_0, 1, -, 1, \to)$ \\
  $p_0$  &    1     & $\sqcup$ &  $(p'_0, 1, -, \sqcup, \gets)$ \\
  $p'_0$ &    1     &    0     &  $(s_0, 1, \to, 0, -)$ \\
  $p'_0$ &    1     &    1     &  $(s_1, 1, \to, 1, -)$ \\
  $p_1$  &    1     &    0     &  $(p_0, 1, -, 1, \to)$ \\
  $p_1$  &    1     &    1     &  $(p_1, 1, -, 0, \gets)$ \\
  $p_1$  &    1     & $\triangleright$ &  $(p_f, 1, -, \sqcup, \to)$ \\
  $p_f$  &    1     &    0     &  $(p'_f, 1, -, 1, \to)$ \\
  $p'_f$ &    1     &    0     &  $(p'_f, 1, -, 0, \to)$ \\
  $p'_f$ &    1     &    1     &  $(p'_f, 1, -, 1, \to)$ \\
  $p'_f$ &    1     & $\sqcup$ &  $(s_0, 1, \to, 0, -)$ \\

  \hline
\end{tabular}
\end{center}
\caption{Problem 1-a: 2-string Turing machine for length
  counting \label{tab:p1_a}}  
\end{table}

b.\\
Using amortized analysis, we can show that for any input string with
length $n$ and its any digit$k$ $(0 \le m \le \log n)$, the total time
the m$th$ digit get flipped is $ 1 / 2^k \cdot n$. And the running
time is

\begin{eqnarray}
  T(n) &=&    \sum_{k = 0}^{\log n} {\frac {1}{2^k} \cdot n \cdot k}
              \nonumber \\
       &=&  n \sum_{k = 0}^{\log n} {\frac {1}{2^k} \cdot k} 
              \nonumber \\
       &\leq& n (1 + \sum_{k = 0}^{\infty} {\frac {1}{2^k} \cdot k}) 
              \nonumber \\
       &\leq& n (1 + 2) \nonumber \\
       &=&    O(n)
\end{eqnarray}

So the time complexity of the TM is $O(n)$.

\section{Problem 2}

\section{Problem 3}

\section{Problem 4}


\appendix
\appendixpage
\addappheadtotoc

\end{document}
