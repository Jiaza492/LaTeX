\documentclass[12pt]{article}

\parindent=.25in
\setlength{\oddsidemargin}{0pt}
\setlength{\textwidth}{440pt}
\setlength{\topmargin}{0in}

\usepackage{amsmath}
\usepackage[dvips]{graphicx}
\usepackage{verbatim}
\usepackage{appendix}

\title{COMS 4236 Homework 3}
\author{Mengqi Zong $<mz2326@columbia.edu>$}

\begin{document}

\maketitle

\setlength{\parindent}{0in}

\section*{Problem 1}

a)

Since C has 20 input variables, there are $2^{20}$ different input
assignments, which is a constant. For any given CIRCUIT-SAT-20
problem, we can solve it using brutal force algorithm in
$O(2^{20})$. That is, CIRCUIT-SAT-20 is in P. \\

b)

$B \in NP$. \\

Since C is NP-complete, given an input assignment $\alpha_C$, we can
verify the output in P. Because $B \le_p C$, for any input assignment
$alpha_b$, there is a poly-time reduction $R_{B \to C}(\alpha_B)$ from B to
C. Therefore, we can also verify the output of $\alpha_B$ in
polynomial time. So $B \in NP$. \\

B is NP-hard. \\

Since $A \le_p B$, there is a poly-time reduction $R_{A \to
  B}(\alpha_A)$ from A to B. Because A is NP-complete, A is
NP-hard. If $B \in P$, then we can also solve A in poly-time, which 
contradicts the fact that A is NP-hard. So, B must be NP-hard. \\

To sum up, B is NP-complete.

\section*{Problem 2}

a)

$\bar {L} \in coC$. \\

Since L is complete for class C, $L \in C$. By the definition of $coC
= \{ \bar {L} | L \in C \}$, we know that $\bar {L} \in coC$. \\

$\bar {L}$ is coC-hard. \\

Since L is complete for class C, then L is C-hard. That is, any
language $L' \in C$ can be reduced to L. For any input assignment
$\alpha \in L'$, we have $R(\alpha) \in L$. Then for any input
assignment $\alpha \in \bar {L'}$, we have $R(\alpha) \in \bar
{L}$. That is, any language $\bar {L'} \in coC$ can be reduced to $\bar
{L'}$. So $\bar {L}$ is coC-hard. \\ 

To sum up, $\bar {L}$ is coC-complete. \\

b)

If C = coC, then M is in coC. \\

Since M is complete for C, $M \in C$. Due to C = coC, then $M \in
coC$. \\

If M is in coC, then C = coC. \\

I will prove that coC is closed. Since C is closed, then whenever L
reduces to L' and L' is in C, then also L is in C. We can also get
whenever $\bar {L}$ reduces to $\bar {L'}$ and $\bar {L'}$ is in coC,
then also $\bar {L}$ is in C. So coC is also closed. \\

Since M is C-complete, any language L in C can be reduced to M. Since
M is in coC and coC is closed, then $L \in coC$. We get $C \subset
coC$. \\

Since M is in coC, then $\bar {M} \in C$. Form part a, we know that
$\bar {M}$ is coC-complete. That means any language L in coC can be
reduced to $\bar {M}$. Because $\bar {M}$ is in C and C is closed,
then $L \in C$. We get $coC \subset C$.

To sum up, C = coC.

\section*{Problem 3}

a)

NP \\

For any language L that $L \le_p L'$ and $L' \in NP$, we will prove
that L is in NP. \\

L can be verified in polynomial time. Since L' is in NP, we know that
L' can be verified in polynomial time. We also know the reduction
takes poly-time, then we can verify L in polynomial time by first
reducing the input $\alpha$ into $R_{L \to L'}(\alpha)$, then verify
$R_{L \to L'}(\alpha)$. So L is in NP. \\

coNP \\

For any language L that $L \le_p L'$ and $L' \in coNP$, we will prove
that L is in coNP. \\

Since L' is in coNP, $\bar {L'} \in NP$. And $\bar {L'}$ can be
verified in polynomial time. Since the reduction is polynomial time,
then $\bar {L}$ can also be verified in polynomial time. So $\bar {L}
\in NP$, and L is in coNP. \\

PSPACE \\

For any language L that $L \le_p L'$ and $L' \in PSPACE$, we will
prove that L is in PSPACE. \\

We know that $TIME(f(n)) \subset NTIME(f(n)) \subset SPACE(f(n))$, so
$PTIME \subset PSPACE$. Since L' is in PSPACE and the reduction takes
at most PSPACE, then L is in PSPACE. \\

b)

There exists language L and L' that $L \le_p L'$ and $L' \in
SPACE(n)$, but L is not in SPACE(n). \\

Let language L is a language in $TIME(n^2)$ and $L' = pad(L,
n^2)$. From homework 2, we know that L' is in TIME(n). Since
$TIME(f(n)) \subset NTIME(f(n)) \subset SPACE(f(n))$, we know that L'
is in SPACE(n). And obviously, the reduction from L to L' takes
$TIME(n^2)$, which is polynomial. But L is not necessarily in
SPACE(n). Because $TIME(n^2) \subset SPACE(n^2)$, L could be in
$SPACE(n^2)$. So SPACE(n) is not closed under polynomial time
reductions.

\section*{Problem 4}

a)

DAG Reachability problem is in NL. \\

Given $H = (N,A)$, node s, t, we can check if there is a path in H
from s to t in NL. All we need to do is start at node s, then randomly
guess the next node to visit, until we reach node t or there's no node
to visit (the graph is acyclic). Since we just need to store the
current node, which takes log-space, then DAG Reachability problem is
in NL. \\

DAG Reachability problem is NL-hard.

We know that general Graph Reachability problem is NL-complete. And we
can reduce Graph Reachability problem to a DAG Reachability
problem. The reduction is the transformation given in the
content. Since we just need to record the index of each node, and the
respective number i,j, which takes log space, this transformation is
log-space. As a result, DAG Reachability problem is NL-hard. \\

To sum up, DAG Reachability is NL-complete. \\

b)

Graph Cyclicity is in NL. \\

We can solve the Graph Cyclicity problem in NL. All we need to do
guess a node which is in the cycle. Then guess the length of the
cycle. Then start traversing from this node, guess the possible node
to visit. Till the length of the cycle step, if the current node is
the starting node, then return true. If not, return false. Since we
just need to store the starting node, the current node and the length
of the circle, which takes log-space, Graph Cyclicity is in NL. \\

Graph Cyclicity is NL-hard. \\

We can reduce the DAG Graph Reachability problem to Graph
Cyclicity. Given $H = (N,A)$, two node s, t, we can translate graph H
into another DAG $H' = (N, A'), A' = A \cup {(t,s)}$. Obviously, the
reduction takes log-space. Since H is a DAG, then the only possible
circle in H' occurs when there's a path in H from s to t. As a result,
Graph Cyclicity is NL-hard. \\

To sum up, Graph Cyclicity is NL-complete. \\

c)

As we can see, Graph Acyclicity is the complement problem of Graph
Cyclicity. Since Graph Cyclicity is NL-complete, from problem 2-a, we
get that Graph Acyclicity is coNL-complete. \\

From class we know that NL=coNL. Since Graph Acyclicity is
coNL-complete, then Graph Acyclicity is NL-complete. \\

\section*{Problem 5}

a)

First, we can easily verify that $a = b \land c$ that satisfies the
inequalities. \\

Second, we will prove that the value of a that satisfies the
inequalities is unique. \\

From $a \le b$ and $a \le c$, we get $2a \le b + c$. Since $b + c \le a
+ 1$, we get $2a \le b + c \le a + 1$. That is, $a \le 1$. Since $0 \le
a$, we get $0 \le a \le 1$. \\

Here are all the possible cases:

\begin{enumerate}
\item Both b and c are 1. \\
  In this case, from $b + c \le a + 1$, we get $1 \le a$. Since $0 \le
  a \le 1$, we know that $a = 1$ is the unique value that satisfies
  the inequalities.
\item At least of b and c is 0. \\
  In this case, from $a \le b$ and $a \le c$, we know that $a \le
  0$. Since $0 \le a \le 1$, we know that $a = 0$ is the unique value
  that satisfies the inequalities.
\end{enumerate}

To sum up, the value $a = b \land c$ is the unique value over real
numbers that satisfies the inequalities. \\

b)

Here is the set of inequalities: $b \le a, c \le a, a \le b + c, a \le
1$. \\

First, we can easily verify that $a = b \lor c$ that satisfies the
inequalities. \\

Second, we will prove that the value of a that satisfies the
inequalities is unique. \\

From $b \le a$ and $c \le a$, we get $b + c \le 2a$. Since $a \le b +
c$, we get $a \le b + c \le 2a$. That is, $0 \le a$. Since $a \le 1$,
we get $0 \le a \le 1$. \\

Here are all the possible cases:

\begin{enumerate}
\item Both b and c are 0. \\
  In this case, from $a \le b + c$, we get $a \le 0$. Since $0 \le
  a \le 1$, we know that $a = 0$ is the unique value that satisfies
  the inequalities.
\item At least of b and c is 1. \\
  In this case, from $b \le a$ and $c \le a$, we know that $1 \le
  a$. Since $0 \le a \le 1$, we know that $a = 1$ is the unique value
  that satisfies the inequalities.
\end{enumerate}

To sum up, the value $a = b \lor c$ is the unique value over real
numbers that satisfies the inequalities. \\

c)

We can reduce the Circuit Value problem with fan-in 2 to a Linear
Inequalities problem. \\

Here is the reduction: 

\begin{itemize}
\item For any AND gate $a = b \land c$ in circuit C, we construct the
  following inequalities: $a le b, a \le c, b + c le a + 1, 0 le a$.
\item For any OR gate $a = b \lor c$ in circuit C, we construct the
  following inequalities: $b \le a, c \le a, a \le b + c, a \le 1$.
\end{itemize}

Obviously, this reduction takes log-space. And since Circuit Value
problem with fan-in 2 is a P-complete problem, Linear Inequalities
problem is P-hard under log-space reductions.

\end{document}
