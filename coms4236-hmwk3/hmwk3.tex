\documentclass[12pt]{article}

\parindent=.25in
\setlength{\oddsidemargin}{0pt}
\setlength{\textwidth}{440pt}
\setlength{\topmargin}{0in}

\usepackage{amsmath}
\usepackage[dvips]{graphicx}
\usepackage{verbatim}
\usepackage{appendix}

\title{COMS 4236 Homework 3}
\author{Mengqi Zong $<mz2326@columbia.edu>$}

\begin{document}

\maketitle

\setlength{\parindent}{0in}

\section*{Problem 1}

a) \\



b) \\

B is NP. \\

Since C is NP-complete, given an input assignment $\alpha_C$, we can
verify the output in P. Because $B \le_p C$, for any input assignment
$alpha_b$, there is a poly-time reduction $R_{B \to C}(\alpha_B)$ from B to
C. Therefore, we can also verify the output of $\alpha_B$ in
polynomial time. So B is NP. \\

B is NP-hard. \\

Since $A \le_p B$, there is a poly-time reduction $R_{A \to
  B}(\alpha_A)$ from A to B. Because A is NP-complete, A is
NP-hard. If B is P, then we can also solve A in poly-time, which 
contradicts the fact that A is NP-hard. So, B must be NP-hard. \\

To sum up, B is NP-complete.

\section*{Problem 2}

a) \\

$\bar {L}$ is coC. \\

Since L is complete for class C, $L \in C$. By the definition of $coC
= \{ \bar {L} | L \in C \}$, we know that $\bar {L}$ is coC. \\

$\bar {L}$ is coC-hard. \\

Since L is complete for class C, then L is C-hard. That is, any
language $L' \in C$ can be reduced to L. For any input assignment
$\alpha \in L'$, we have $R(\alpha) \in L$. Then for any input
assignment $\alpha \in \bar {L'}$, we have $R(\alpha) \in \bar
{L}$. That is, any language $\bar {L'} \in coC$ can be reduced to $\bar
{L'}$. So $\bar {L}$ is coC-hard. \\ 

To sum up, $\bar {L}$ is coC-complete. \\

b) \\





\section*{Problem 3}



\end{document}
