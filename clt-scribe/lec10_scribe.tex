\documentclass[12pt]{article}
\parindent=.25in

\setlength{\oddsidemargin}{0pt}
\setlength{\textwidth}{440pt}
\setlength{\topmargin}{0in}
\usepackage{amssymb}
\usepackage{amsfonts}
\usepackage{amsmath}
\usepackage{latexsym}
\usepackage[center]{subfigure}
\usepackage{epsfig}
\usepackage{hyperref}

\usepackage{amsthm}

\newtheorem{theorem}{Theorem}
\newtheorem{lemma}[theorem]{Lemma}
\newtheorem{remark}{Remark}
\newtheorem{fact}[theorem]{Fact}
\newtheorem{definition}[theorem]{Definition}
\newtheorem{corollary}[theorem]{Corollary}
\newtheorem{proposition}[theorem]{Proposition}
\newtheorem{claim}[theorem]{Claim}
\newtheorem{conjecture}[theorem]{Conjecture}
\newtheorem{observation}[theorem]{Observation}
\newtheorem{assumption}[theorem]{Assumption}
\newtheorem{example}[theorem]{Example}

\newcommand{\noi}{{\noindent}}
\newcommand{\ms}{{\medskip}}
\newcommand{\msni}{{\medskip \noindent}}

\newcommand{\la}{\langle}
\newcommand{\ra}{\rangle}
\newcommand{\calc}{{\cal C}}
\newcommand{\cald}{{\cal D}}
\newcommand{\calh}{{\cal H}}
\newcommand{\cala}{{\cal A}}

\newcommand{\sign}{\mathrm{sign}}
\newcommand{\eps}{\epsilon} 
\newcommand{\poly}{\mathrm{poly}}
\newcommand{\size}{\mathrm{size}}
\newcommand{\depth}{\mathrm{depth}} 

\pagestyle{headings}    % Go for customized headings

\newcommand{\handout}[5]{
   \noindent
   \begin{center}
   \framebox{
      \vbox{
    \parbox[t]{4in} {\bf #1 } \vspace{3mm}  {\hfill \bf #2 }
       \vspace{2mm}
       \hbox to 6.00in { {\Large \hfill #5  \hfill} }
       \vspace{1mm}
       \hbox to 6.00in { {\it #3 \hfill #4} }
      }
   }
   \end{center}
   \vspace*{1mm}
}

\begin{document}

\setlength{\parindent}{0in}

\handout{COMS 6253: Advanced Computational Learning Theory}{Spring 2012}
{Lecturer: Rocco Servedio}
{Scribe: Mengqi Zong}{Lecture 10: March 29, 2012}

\thispagestyle{plain}

\section{Last Time and Today}

{\bf Previously:}
\begin{itemize}
\item Use $NS_{\epsilon}$ to get Fourier Concentration
\item Proved Peres theorem on $NS_{\epsilon}$ (halfspaces)
\item Uniform distribution learning beyond LDA
  \begin{itemize}
  \item LTFs, the ``Chow parameters''
  \item ``random'' DNFs
  \end{itemize}
\end{itemize}

{\noindent \bf Today:}
\begin{itemize}
\item Learning r-juntas.
\end{itemize}

{\noindent \bf Relevant Readings:} 

\begin{itemize}
\item Elchanan Mossel, Ryan O'Donnell, Rocco A. Servedio
  (2003). Learning Juntas
\item T. Siegenthaler (1984). Correlation-Immunity of Nonlinear
  Combining Functions for Cryptographic Applications
\item Gregory Valiant (2012). Finding Correlations in
    Subquadratic Time, with Applications to Learning Parities and
    Juntas with Noise
\end{itemize}

\section{Introduction}

\begin{definition}
Variable $i$ in $f(x_1,...,x_n)$ is relevant if $\; \exists x \in
\{-1, 1 \}$ such that $f(x^{i \leftarrow 1}) \neq f(x^{i \leftarrow
  -1})$.
\end{definition}

\begin{definition}
Function $f: \{ -1, 1 \}^n \rightarrow \{ -1,1 \}$ is an r-junata if
$f$ has at  $k \le r$ relevant variables.
\end{definition}

\begin{example}
$x_{17} \oplus (x_{412} \lor (x_{916,774} \oplus x_{17}))$ is a
3-junta. This can be treated as a function of the 3 variables $x_{17},
x_{412}, x_{916,774}$.
\end{example}

\begin{observation}
Let $f$ be an r-junta. Then $f$ has a DT of size $2^r$ and a DNF of $s
\le 2^r$ terms.
\end{observation}

So to learn w(1)-size DNFs, DTs in $\poly(n)$ time, we need to be able
to learn juntas. \\

Let $\calc_r = \{\text {all r-juntas over n variables} \}$. By
learning r-juntas, we mean learn any r-junta over the n variables. So
it's very natural to ask: What can we hope for with respect to
learning $\calc_r$? \\

\section{Description Length}

We first talk about the description length of an r-junta. The
description length of an r-junta is $r \log n + 2^r$ bits. First,
since there are $n$ variables, we need $\log n$ bits to represent each
variable. Then for $r$ variables, we need $r \log n$ bits. Second,
for $r$ variables, the truth table contains $2^r$ entries. So it takes
$2^r$ bits to write down the truth table.  To sum up, the description
length of an r-junta is $r \log n + 2^r$. \\

Note that all the $r$ variables in an r-junta are relevant, this means
that $2^r$ is the minimum number of examples we could hope for. So, the
running time of the learning algorithm for r-juntas is associated 
with $n$ and $2^r$. It is unknown whether we can learn r-juntas in
time $\poly (n, 2^r)$. \\

\section{Observations and Approaches}

\begin{observation}
\begin{equation*}
|\calc_r| \le n^r \cdot 2^{2^r}
\end{equation*}
\end{observation}

This observation follows by the fact that there are $\binom {n}{r}$
possible ways to choose $r$ relevant variables from $n$ variables and
there are $2^{2^r}$ possible truth tables over $r$ variables. \\

From this observation, we know that we can learn an r-junta in $O(n^r
\cdot 2^{2^r})$ by trying every possible concept in $\calc_r$. Due
to the $2^{2^r}$ part, this is not an efficient algorithm.

\begin{observation}
To learn r-juntas, it suffice to be able to find relevant
variables. Given the $r$ relevant variables, $O(r \cdot 2^r)$ examples
suffice to fill in the truth table.
\end{observation}

We can get $r \cdot 2^r$ by applying the method of coupon collector's
problem:

\begin{eqnarray*}
E(T) = k \cdot H_k = k \ln k = 2^r \cdot \ln (2^r) = r \cdot 2^r
\end{eqnarray*}

Now the main focus of learning r-juntas turns to how to find the $r$
relevant variables efficiently.

\begin{observation}
If variable $i$ is irrelevant in f, then $Inf_i(f) = 0$ and $\hat
{f}(S) = 0 \; \forall S \ni i$. If variable $i$ is relevant in r-junta
f, then $Inf_i(f) \ge \frac {1}{2^r}$ and some $S \ni i$ must have
$|\hat {f}(S)| \ge \frac {1}{2^r}$.
\end{observation}

The first part of the observation can be easily verified by the
definition of influence and relevant. We now just prove ``If $i$ is
relevant to f, then some $S \ni i$ must have $|\hat {f}(S)| \ge \frac
{1}{2^r}$''. \\

Recall from previous lectures, we have

\begin{eqnarray*}
Inf_i(f) = \sum_{S \ni i} \hat {f}(S)^2
\end{eqnarray*}

Then for relevant variable $i$, if no $S \ni i$ has $|\hat {f}(S)| \ge
\frac {1}{2^r}$, we get

\begin{eqnarray*}
Inf_i(f)
&=& \sum_{S \ni i} \hat {f}(S)^2 \\
&=& 2^r \cdot \hat {f}(S)^2 \\
&<& 2^r \cdot (\frac{1}{2^r})^2 \\
&<& 2^r
\end{eqnarray*}

This contradicts with the fact that $Inf_i(f) \ge \frac{1}{2^r}$. So
some $S \ni i$ must have $|\hat {f}(S)| \ge \frac {1}{2^r}$. \\

Note that from the first part of the observation, we know if $S$ has
irrelevant variables, $\hat {f}(S) = 0$. So, there are at most 
$2^r \; S$ whose $\hat {f}(S) \neq 0$. So we get the second
equation. \\

From this observation, we know that we can learn concepts from
$\calc_r$ in $n^r \cdot poly(2^r)$. Here is the algorithm:

\begin{itemize}
\item For all S with $1 \le |S| \le r$, we estimate $\hat {f}(S)$
  with accuracy of $\pm 0.1 \cdot \frac {1}{2^r}$. Whenever we find
  a $S$ with $\hat {f}(S) \neq 0$, add all variables in $S$ to the
  collection of relevant variables.
\item After we get all relevant variables by this way, we can learn
  the r-junta with $O(r \cdot 2^r)$ more examples.
\end{itemize}

Note that there are $n^r$ possible $S$ here. And given a $S$, the
estimation takes $\poly (2^r)$ time. So in total, this algorithm takes
$n^r \cdot \poly (2^r)$ time. Due to the $n^r$ part, this algorithm is
still not good enough.

\section{Main Result}

We can learn $\calc_r$ in $n^{0.704 \cdot r} \cdot \poly(2^r)$
time. This result is shown in the paper ``Learning Juntas'' by
Elchanan Mossel, Ryan O'Donnell and Rocco Servedio in 2003. \\

Note that the state of the art result for learning r-juntas is
$n^{0.61 \cdot r} \cdot \poly(2^r)$ by Greg Valiant. 

\subsection{High level idea of the method}

We will look at 2 different polynomial representations of $f$:

\begin{itemize}
\item Fourier representation
\item GF(2) representation
\end{itemize}

The intuition of the method is if $f$ is ``bad'' for Fourier-based
learning, i.e. all its non-constant Fourier coefficients are on
high-degree monomials, then $f$ must be ``good'' for GF(2)-based
learning.

\subsection{Find 1 relevant variable efficiently is enough}

In the previous observation, we know that the most difficult part of
learning juntas is how to find $r$ relevant variables efficiently. We
now show to learn r-juntas efficiently, all we need to do is to find 1
relevant variable efficiently.

\begin{claim}
Suppose A is a T(n,r)-time algorithm which finds a relevant variable
in an r-junta, given uniform (x, f(x)) random examples. Then there's
an algorithm to learn r-juntas running in $T(n,r) \cdot \poly (n,
2^r)$ time.
\end{claim}

\begin{proof}
We will use A to get relevant variable ``i'', or find out no variable
is relevant. If no variable is relevant, then done. \\

Now we talk about the situation that function is not a constant
function. Then, we will run A in a recursive style. It's like a binary
search:

\begin{enumerate}
\item Run A to get a relevant variable i.
\item Run A on $(x, f(x)) |_{x_i = 1}$.
\item Run A on $(x, f(x)) |_{x_i = -1}$.
\end{enumerate}

In this way, we build a decision tree with depth $\le r$. As to the
running time, we have

\begin{eqnarray*}
T(r)
&\le& 2T(r-1) + T(n, r) \cdot \poly (n) \\
&\le& 4T(r-2) + 2T(n, r-1) \cdot \poly (n) + T(n, r) \cdot \poly (n) \\
&\le& (2^rT(n, 1) + 2^{r-1}T(n, 2) + ... 2T(n, r-1) + T(n, r)) \cdot
\poly (n) \\
&\le& (2^r + 2^{r-1} + ... + 2 + 1) \cdot T(n, r) \cdot \poly (n) \\
&\le& (2^{r+1} - 1) \cdot T(n, r) \cdot \poly (n) \\
&\le& T(n, r) \cdot \poly (n, 2^r) \\
\end{eqnarray*}

The running time of the algorithm to learn r-juntas is $T(n,r) \cdot
\poly (n, 2^r)$. \\
\end{proof}

\subsection{Fourier-based learning}

\begin{fact}
If $f$ has $\hat {f}(S) \neq 0$ for some $S$ with $1 \le |S| \le c
\cdot r$, then we can find  a relevant variable in $\le 2^r \cdot
n^{cr}$ time.
\end{fact}

From the previous observation, we know that if $\hat {f}(S) \neq 0$,
then all variables in $S$ is relevant. To find such a $S$, we simply
try every possible S with $1 \le |S| \le c \cdot r$. There are $\le
2^r \cdot n^{cr}$ different $S$ in total. \\

Later, we'll see if $f$ has no such $S$ as stated in this fact, then
a GF(2)-based learning algorithm will work.

\subsection{GF(2)-based learning}

GF(2) is the Galois field of two elements. It is the smallest finite
field. The GF(2) representation of $f$ a multilinear polynomial over
the filed $GF(2) = \{ 0, 1 \}$, and all math is done with modulo
2. This ensures that the field is closed under addition and
multiplication. In GF(2), 0 is equivalent to false, 1 is equivalent to
true, addition is equivalent to parity function, and multiplication is
equivalent to AND function. \\

\begin{fact}
Let $f: \{ 0, 1 \}^n \rightarrow \{ 0, 1\}$. Then $f$ has
\begin{enumerate}
\item a unique representation as a multilinear polynomial $P_R$ over
  real numbers. All coefficients are integers.
\item a unique representation as a GF(2) polynomial $P_{GF(2)}$.
\item If $p_1, p_2$ are 2 degree-d GF(2) polynomials for $f$ and $g$,
  and $f \neq g$, then $Pr_x \[ f(x) \neq g(x) \] \ge \frac
  {1}{2^d}$.
\end{enumerate}

\end{fact}

The proof is left as the official homework. \\

An easy relation between $P_R$ and $P_{GF(2)}$: we can get $P_{GF(2)}$
from $P_R$ by reducing all coefficients mod 2.

\begin{eqnarray*}
P_R = \sum_{S \subseteq \[ n \]}
{C_S X_S}
\Rightarrow P_{GF(2)} = \sum_{S \subseteq \[ n \]} (C_s \mod 2) X_S
\end{eqnarray*}

So $deg(P_{GF(2)}) \le deg(P_R)$ for all $f$. \\

\begin{example}

\end{example}



\end{document}
