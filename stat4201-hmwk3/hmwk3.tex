\documentclass[12pt]{article}

\usepackage{amsmath}
\usepackage[dvips]{graphicx}
\usepackage{verbatim}
\usepackage{appendix}

\title{Stat 4201 Homework 3}
\author{Mengqi Zong $<mz2326@columbia.edu>$}

\begin{document}

\maketitle

\section*{Problem 1}

I use Fisher's Exact Test to do the data analysis. Here is the result
from R:

\begin{verbatim}

	Fisher's Exact Test for Count Data

data:  matrix(c(149, 48, 129, 68), 2, 2) 
p-value = 0.03547
alternative hypothesis: true odds ratio is not equal to 1 
95 percent confidence interval:
 1.032059 2.602066 
sample estimates:
odds ratio 
  1.634216 

\end{verbatim}

The proportion of incorrect recollections among the left-handed
$(75.6\%)$ was higher than the correspond proportion among the
right-handed. And the data are not consistent with the hypothesis of
equal proportions of incorrect recollections in the populations of
left-handed and righ-handed people ($two-sided p-value = .035$). That
is, the data indicate that left- or right-handedness is associated
with correct reollection of the orientation. And the $95\%$ confidence
interval for the odds ratio is $[1.032, 2.602]$. This means that
left-handed people is $1.032$ times to $2.602$ times higher than
right-handed people to recollect the comet orientation incorrectly.

\section*{Problem 2}

The Simpson's paradox is caused by the different proportion of math
and physics majors. In this example, the paradox is caused by Eagle
College's high proportion of Mathematics majors, which has a higher
pass rate than that of both colleges' Physics majors.\\

Among math students, $67\%$ passed at Crane, compared to $56\%$ at
Eagle. Among physics students, $36\%$ passed at Crane, compared to
$25\%$ at Eagle. Among all students (ignoring department), $50\%$
passed at Crane compared at $53\%$ at Eagle. Intuitively, both
college's average pass rate are affected by the poor pass rate of the
physics majors severely. However, due to Eagle college's low 
proportion of physics majors, its average pass rate doesn't ``drop"
from the original math major pass rate much (from $56\%$ to
$53\%$). On the contrary, Crane's does (from $67\%$ to $50\%$). As a
result, Simpson's paradox arises.\\

However, Crane still won, though its average pass rate is less than
that of Eagle's. Because Crane won both match-ups involving students
with supposedly comparable training. And what average pass rate did,
trying to compare pass between different majors which are not
comparable, is not reasonable.\\

\noindent * To answer this question, I referenced the solution at page
578 of Ramsey and Schafer $2^{nd}$ edition.

\section*{Problem 3}

I use the Mantel-Haenszel test to do the data analysis. Here is the
result from R:

\begin{verbatim}

	Mantel-Haenszel chi-squared test with continuity correction

data:  data.p3 
Mantel-Haenszel X-squared = 17.4618, df = 1, p-value = 2.931e-05
alternative hypothesis: true common odds ratio is not equal to 1 
95 percent confidence interval:
 0.1296716 0.4806826 
sample estimates:
common odds ratio 
        0.2496615 

\end{verbatim}

From the analysis, we can see that the odds of a tire-related fatal
accident depend on whether the sports utility vehicle is a Ford
($p-value = 2.931e-05$). And the $95\%$ confidence interval for odds
ratio between Ford and other is $[0.1297, 0.4807]$. Also, the common
odds ratio is 0.2497. This means that Ford's sports utility vehicles
will cause 4 ($1/0.2497 \approx 4$) times tire accidents than other
brands' vehicles.

\appendix
\appendixpage
\addappheadtotoc

The code is listed below:

\verbatiminput{hmwk3.r}

\end{document}
