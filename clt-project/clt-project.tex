\documentclass[12pt]{article}
\parindent=.25in

\setlength{\oddsidemargin}{0pt}
\setlength{\textwidth}{440pt}
\setlength{\topmargin}{0in}
\usepackage{amssymb}
\usepackage{amsfonts}
\usepackage{amsmath}
\usepackage{latexsym}
\usepackage[center]{subfigure}
\usepackage{epsfig}
\usepackage{hyperref}

\usepackage{amsthm}

\newtheorem{theorem}{Theorem}
\newtheorem{lemma}[theorem]{Lemma}
\newtheorem{remark}{Remark}
\newtheorem{fact}[theorem]{Fact}
\newtheorem{definition}[theorem]{Definition}
\newtheorem{corollary}[theorem]{Corollary}
\newtheorem{proposition}[theorem]{Proposition}
\newtheorem{claim}[theorem]{Claim}
\newtheorem{conjecture}[theorem]{Conjecture}
\newtheorem{observation}[theorem]{Observation}
\newtheorem{assumption}[theorem]{Assumption}
\newtheorem{example}[theorem]{Example}

\newcommand{\noi}{{\noindent}}
\newcommand{\ms}{{\medskip}}
\newcommand{\msni}{{\medskip \noindent}}

\newcommand{\la}{\langle}
\newcommand{\ra}{\rangle}
\newcommand{\calc}{{\cal C}}
\newcommand{\cald}{{\cal D}}
\newcommand{\calh}{{\cal H}}
\newcommand{\cala}{{\cal A}}

\newcommand{\sign}{\mathrm{sign}}
\newcommand{\eps}{\epsilon} 
\newcommand{\poly}{\mathrm{poly}}
\newcommand{\size}{\mathrm{size}}
\newcommand{\depth}{\mathrm{depth}} 

\title{PAC-learning, Occam Algorithm and Compression}
\author{Mengqi Zong $<mz2326@columbia.edu>$}

\begin{document}

\maketitle

\setlength{\parindent}{0in}

\thispagestyle{plain}

\section{Introduction}

% talk about the structure of this paper

The probably approximately correct learning model (PAC learning), proposed in 1984 by Leslie Valiant, is a framework for mathematical analysis of machine learning and it has been widely used to investigate the phenomenon of learning from examples. \\

In COMS 6253, we spent most of our time studying the results based on PAC-learning model. So it is worthy for us to take a closer look at this model. \\

This paper consists of three parts. In the first part, we will talk about PAC-learning, Occam algorithm and the relationship between the two. In the second part, we will talk about the relationship between PAC-learnability and data compression. In the last part, we attempt to apply PAC-learning model to analyze data compression, particularly, relationship between learning r-junta and lossless compression is investigated. \\

This entire paper will focus on learning in discrete domains. 

% have a \emph {notes} flavor.

\section{Part I: PAC-learning and Occam algorithm}

\subsection{PAC-learning}

% put some definitions here; talk about the algorithm

What is the PAC-learning model? Generally, it is a model for mathematical analysis of machine learning, particularly, for the phenomenon of learning from examples. \\

In this model, the target concept we want to learn is a subset of a domain of elements, and a concept class is a set of such concepts. A learning algorithm is presented with a collection of examples, each example is a labeled domain element indicating whether or not it is a member of the target concept in the class. All these examples, the domain elements, are drawn randomly according to a fixed probability distribution. Informally, the concept class is PAC-learnable if the learning algorithm runs in polynomial time and, with high probability, outputs the description of a concept in the class that differs by only a small amount from the unknown concept.

\subsubsection{Preliminary notation and definitions}

If $S$ and $T$ are sets, then we denote the symmetric difference of $S$ and $T$ by $S \oplus T = (S - T) \cup (T - S)$. The cardinality of a set $S$ is denoted by $|S|$. If $\Sigma$ is a alphabet, then $\Sigma^*$ denotes the set of all finite length strings of elements of $\Sigma$. If $w \in \Sigma^*$, then the length of $w$, denoted by $|w|$, is the number of symbols in the string $w$. Let $\Sigma^{[n]}$ denote the set $\{ w \in \Sigma^*: |w| \le n \}$. \\

Define a \emph {concept class} to be a pair $C = (C,X)$, where $X$ is a set and $C \subseteq 2^X$. The domain of $C$ is $X$, and the elements of $C$ are \emph {concepts}. \\

We describe a context for representing concepts over $X$. Define a class of representations to be a four-tuple ${\bf R} = (R, \Gamma, c, \Sigma)$. $\Sigma$ and $\Gamma$ are sets of characters. Strings composed of characters in $\Sigma$ are used to describe elements of $X$, and strings of characters in $\Gamma$ are used to describe concepts. $R \subseteq \Gamma^*$ is the set of strings that are concept descriptions or \emph {representations}. Let $c: R \rightarrow 2^{\Sigma^*}$ be a function that maps these representations into concepts over $\Sigma$. \\

Note that if ${\bf R} = (R, \Gamma, c, \Sigma)$ is a class of representations, then there is an associated concept class ${\bf C(R)} = (c(R), \Sigma^*)$, where $c(R) = \{ c(r): r \in R \}$. \\

We write $r(x) = 1$ if $x \in c(r)$, and $r(x) = 0$ if $x \notin c(r)$. An \emph {example} of $r$ is a pair $(x, r(x))$. The length of an example $(x, r(x))$ is $|x|$. A sample of size $m$ of concept $r$ is a multiset of $m$ examples of $r$. \\

For a class of representations, the \emph {membership problem} is that of determining, given $r \in R$ and $x \in \Sigma^*$, whether or not $x \in c(r)$. For practical reasons, we only consider classes of representations for which the membership problem is decidable in polynomial time. That is, we only consider representation classes ${\bf R} = (R, \Gamma, c, \Sigma)$ for which there exists a polynomial-time algorithm EVAL such that for all $r \in R$, $x \in \Sigma^*$, $\text {EVAL}(r, x) = r(x)$. EVAL runs in time polynomial in $|r|$ and $|x|$. \\

We let $R^{[s]}$ denote the set $\{ r \in R: |r| \le s\}$; that is, the set of all representations from $R$ of length at most $s$. If ${\bf R} = (R, \Gamma, c, \Sigma)$ is a class of representations, $r \in R$, and $\cald$ is a probability distribution on $\Sigma^*$, then $\text{EXAMPLE}(\cald, r)$ is an oracle that, when called, randomly chooses an $x \in \Sigma^*$ according to distribution $\cald$ and returns the pair $(x, r(x))$. \\

A randomized algorithm is an algorithm that behaves like a deterministic one with the additional property that, at one or more steps during its execution, the algorithm can flip a fair two-sided coin and use the result of the coin flip in its ensuing computation.

\subsubsection{PAC-learnability}

\begin{definition}
The representation class ${\bf R} = (R, \Gamma, c, \Sigma)$ is PAC-\emph {learnable} if there exists a (possibly randomized) algorithm L and a polynomial $p_L$ such that for all $s, n \ge 1$, for all $\epsilon$ and $\delta$, $0 < \epsilon, \delta < 1$, for all $r \in R^{[s]}$, and for all probability distributions $\cald$ on $\Sigma^{[n]}$, if L is given as input the parameters $s, \epsilon$, and $\delta$, and may access the oracle $\text {EXAMPLE}(\cald, r)$, then L halts in time $p_L(n, s, 1/\epsilon, 1/\delta)$ and, with probability at least $1 - \delta$, outputs a representation $r' \in R$ such that $\cald (r' \oplus r) \le \epsilon$. Such an algorithm L is a polynomial-time learning algorithm for {\bf R}.
\end{definition}

Here are few comments about PAC-learning and PAC-learnability:

\begin{enumerate}
  \item 
\end{enumerate}

\subsection{Occam's razor and Occam algorithm}

% historical notes about Occam's Razor; Occam 


\subsection{PAC-learnability and Occam algorithm}




\section{Part II: PAC-learning and data compression}

% the interpretation of PAC-learning as data compression



\section{Part III: PAC-learning model for data compression algorithms: possibilities and difficulties}

% talk about the attempt I tried to use PAC-learning Model to model
% data compression algorithm.





\end{document}
