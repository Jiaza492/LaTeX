\documentclass[12pt]{article}
\parindent=.25in

\setlength{\oddsidemargin}{0pt}
\setlength{\textwidth}{440pt}
\setlength{\topmargin}{0in}
\usepackage{amssymb}
\usepackage{amsfonts}
\usepackage{amsmath}
\usepackage{latexsym}
\usepackage[center]{subfigure}
\usepackage{epsfig}
\usepackage{hyperref}

\usepackage{amsthm}

\newtheorem{theorem}{Theorem}
\newtheorem{lemma}[theorem]{Lemma}
\newtheorem{remark}{Remark}
\newtheorem{fact}[theorem]{Fact}
\newtheorem{definition}[theorem]{Definition}
\newtheorem{corollary}[theorem]{Corollary}
\newtheorem{proposition}[theorem]{Proposition}
\newtheorem{claim}[theorem]{Claim}
\newtheorem{conjecture}[theorem]{Conjecture}
\newtheorem{observation}[theorem]{Observation}
\newtheorem{assumption}[theorem]{Assumption}
\newtheorem{example}[theorem]{Example}

\newcommand{\noi}{{\noindent}}
\newcommand{\ms}{{\medskip}}
\newcommand{\msni}{{\medskip \noindent}}

\newcommand{\la}{\langle}
\newcommand{\ra}{\rangle}
\newcommand{\calc}{{\cal C}}
\newcommand{\cald}{{\cal D}}
\newcommand{\calh}{{\cal H}}
\newcommand{\cala}{{\cal A}}

\newcommand{\sign}{\mathrm{sign}}
\newcommand{\eps}{\epsilon} 
\newcommand{\poly}{\mathrm{poly}}
\newcommand{\size}{\mathrm{size}}
\newcommand{\depth}{\mathrm{depth}} 

\title{PAC-learning, Occam Algorithm and Compression}
\author{Mengqi Zong $<mz2326@columbia.edu>$}

\begin{document}

\maketitle

\setlength{\parindent}{0in}

\thispagestyle{plain}

\section{Introduction}

% talk about the structure of this paper

The probably approximately correct learning model (PAC learning), proposed in 1984 by Leslie Valiant, is a framework for mathematical analysis of machine learning and it has been widely used to investigate the phenomenon of learning from examples. \\

In COMS 6253, we spent most of our time studying the results based on PAC-learning model. So it is worthy for us to take a closer look at this model. \\

This paper consists of three parts. In the first part, we will talk about PAC-learning, Occam algorithm and the relationship between the two. In the second part, we will show that for many classes, PAC-learnability is equivalent to data compression. In the last part, we attempt to apply PAC-learning model to analyze data compression. This entire paper will have a note flavor.

\section{PAC-learning and occam algorithm}

\subsection{PAC-learning}

% put some definiton here; talk about the algorithm



\subsection{Occam's razor and occam algorithm}

% historical notes about Occam's Razor; Occam 


\subsection{PAC-learnability and occam algorithm}




\section{PAC-learning and data compression}

% the interpretation of PAC-learning as data compression



\section{PAC-learning model for data compression algorithms: possibilities and difficulties}

% talk about the attempt I tried to use PAC-learning Model to model
% data compression algorithm.





\end{document}
