\documentclass[12pt]{article}
\parindent=.25in

\setlength{\oddsidemargin}{0pt}
\setlength{\textwidth}{440pt}
\setlength{\topmargin}{0in}

\usepackage{amssymb}
\usepackage{amsfonts}
\usepackage{amsmath}

\DeclareMathOperator*{\argmax}{arg\,max}
\DeclareMathOperator*{\argmin}{arg\,min}

\title{Introduction to Algorithms Note}
\author{Mengqi Zong $<mz2326@columbia.edu>$}

\begin{document}

\maketitle

\setlength{\parindent}{0in}

\section{Growth of Functions}

\subsection{Asymptotic notation}

\begin{eqnarray*}
  \Theta (g(n)) = \{ f(n): && \text{there exist positive constants
    $c_1$, $c_2$, and $n_0$ such that} \\
  && 0 \le c_1 g(n) \le f(n) \text{ for all } n \ge n_0 \} \\
  O(g(n)) = \{ f(n): && \text{there exist positive constants
    $c$ and $n_0$ such that} \\
  && 0 \le f(n) \le cg(n) \text{ for all } n \ge n_0 \} \\
  \Omega (g(n)) = \{ f(n): && \text{there exist positive constants
    $c$ and $n_0$ such that} \\
  && 0 \le cg(n) \le f(n) \text{ for all } n \ge n_0 \}
\end{eqnarray*}

\subsection{Common functions}

\begin{eqnarray*}
  e^x &=& \sum_0^{\infty} \frac {x^i}{i!} \\
  e^x &=& \lim_{n \rightarrow \infty} (1 + \frac {x}{n})^n \\
  n!  &=& \sqrt {2 \pi n} \left(\frac {n}{e} \right)^n 
          \left( 1 + \Theta \left( \frac {1}{n} \right) \right)
\end{eqnarray*}

\section{Recurrences}

\subsection{The master method}

Let $a \ge 1$ and $b \ge 1$ be constants, let $f(n)$ be a function, and let $T(n)$ be defined on the nonnegative integers by recurrence
\begin{equation*}
  T(n) = aT(n/b) + f(n),
\end{equation*}
where we interpret $n/b$ to mean either $\lfloor n/b \rfloor$ or $\lceil n/b \rceil$. Then $T(n)$ can be bounded asymptotically as follows.
\begin{enumerate}
  \item If $f(n) = O(n^{\log_b^a - \epsilon})$ for some constant $\epsilon > 0$, then $T(n) = \Theta (n^{\log_b a})$.
  \item If $f(n) = O(n^{\log_b^a})$, then $T(n) = \Theta (n^{\log_b a} \lg n)$.
  \item If $f(n) = O(n^{\log_b^a + \epsilon})$ for some constant $\epsilon > 0$, and if $af(n/b) \le cf(n)$ for some constant $c < 1$ and all sufficiently large $n$, then $T(n) = \Theta \left( f(n) \right)$.
\end{enumerate}

\end{document}