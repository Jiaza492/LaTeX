\documentclass[12pt]{article}

\parindent=.25in
\setlength{\oddsidemargin}{0pt}
\setlength{\textwidth}{440pt}
\setlength{\topmargin}{0in}

\usepackage{amsmath}
\usepackage[dvips]{graphicx}
\usepackage{verbatim}
\usepackage{appendix}

\title{COMS 4236 Homework 6}
\author{Mengqi Zong $<mz2326@columbia.edu>$}

\begin{document}

\maketitle

\setlength{\parindent}{0in}

\section*{Problem 1}

For any input $x$, let the Boolean expression in disjunctive normal
form in the input be $d$. We first guess a Boolean expression $c$ in
conjunctive normal form that has at most $B$ clauses. Let $a$ be an
assignment to the Boolean expression. Then $(d - c)(a)$ denote that
the value of expression $d - c$ for assignment $a$. We now could show
that the language $L$ defined by this problem is

\begin{eqnarray*}
L = \{ x: \exists \; c \; \forall a \text { such that } (d - c)(a) = 0
\}
\end{eqnarray*}

Since $B$ is given in unary and $c$ has at most $B$ clauses, we can
verify if $(d - c)(a) = 0$ for any $a$ in $poly(x)$ time. Then by the
definition of the certificate version of $\Sigma_2$, we know that $L
\in \Sigma_2$. So this problem is in $\Sigma_2$.

\section*{Problem 2}

1. It's easy to show that DFA Intersection problem is in NPSPACE. We
simply guess a string by first guessing its length and then guessing
every tape cell of the string. Then verify if this string can be
accepted by all $n$ DFAs. Trivially, we can use $n$ work tapes and
each tape initially stores the string $x$. So, this problem is in
NPSAPCE. By Savitch's Theorem, this problem is in PSPACE. \\

2. We will show that the DFA Intersection problem is PSPACE-hard
by reducing LBA problem to this problem. \\

Given an input $x$ to the LBA, we will construct the DFAs as follows:

Let $n = |x| + 2$ (including endmarkers), we construct $n$ DFAs,
$A_1,...,A_n$. Suppose the LBA has a transition $(q, a, i) \rightarrow
(p, b, j)$. Then for all DFA except $A_i, A_j$, they will have a
transition $(a, i) \rightarrow (b, j)$. As to $A_i$, it will have a
transition $(p, a, i) \rightarrow (b, j)$. As to $A_j$, it will have a
transition $(a, i) \rightarrow (q, b, j)$. And when LBA accepts, mark
all DFAs corresponding states accepting. \\

As we can see, all DFAs are different, and LBA accepts $x$ if and only
if all DFAs accepts $x$. So, the DFA intersection problem is
PSPACE-hard. \\

To sum up, the DFA intersection problem is PSPACE-hard.  

\section*{Problem 3}

1. It is trivial that \#2SAT is in \#P. Now we will prove that \#2SAT is
\#P-hard by reducing \#Matchings to \#2SAT. \\

For any \#Matchings problem, given an undirected graph G, we construct
a 2SAT monotone formula as follows:

\begin{itemize}
\item For each edge (i,j) in G, we construct a Boolean variable.
\item For each pair of edges that share the same node, we create a
  new term in the 2SAT.

  For example, suppose there are three edges that connect to node 1:
  (1,2), (1,3), (1,4). And we use Boolean variables $v_2$, $v_3$,
  $v_4$ to represent each edge. Since there are $\binom {3}{2} = 3$
  pairs of edges, we create 3 new terms: $(v_2 \lor v_3)$, $(v_2 \lor
  v_4)$, $(v_3 \lor v_4)$.
\item Combine all terms together with $\land$, we get the \#2SAT
  monotone formula $\phi$.
\end{itemize}

Now we will show that the number of satisfying assignments of $\phi$
equals to the number of matchings in G. For every satisfying
assignment of $\phi$, for each variable $v_i$, if $v_i = 1$, then the
edge represented by $v_i$ is not in the matching. That is, all the
edges whose $v_i = 0$ form a matching of G. We will show that all
those edges don't share the same node. If there is one pair of
edges share the same node, since all edges in the matching is
equivalent to a variable $v_i = 0$, then one term in the \#2SAT will
not be satisfied, then the formula $\phi$ will not be satisfied. This
contradicts with the fact that the assignment is a satisfying
assignment of $\phi$. So, \#2SAT is \#P-hard. \\

To sum up, \#2SAT is \#P-complete. \\

2. It is trivial that \#Node Covers is in \#P. We will prove that
\#Node Covers is \#P-hard by reducing \#2SAT to \#Node Covers. \\

For any \#2SAT problem, given a 2SAT formulas $\phi$, we will create
a undirected graph G as follows:

\begin{itemize}
\item For every variable $v_i$ in $\phi$, we create a node $n_i$ in G.
\item For every term $(v_i \lor v_j)$, we create an edge $(n_i, n_j)$
  in G.
\end{itemize}

Now we will show that the number of node covers of G equals to the
number of satisfying assignment of $\phi$. For every set of nodes that
cover all edges in G, since each edge in G represents a term in the
\#2SAT formula $\phi$, then covering all nodes means all terms in
$\phi$ are satisfied. So, each node cover of G is equivalent to a
satisfying assignment to $\phi$. And each nodes in the node cover
means the variable represented by this node is 1. So, \#Node Covers is
\#P-hard. \\

To sum up, \#Node Cover is \#P-complete.

\end{document}
