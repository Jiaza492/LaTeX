\documentclass[12pt]{article}
\parindent=.25in

\setlength{\oddsidemargin}{0pt}
\setlength{\textwidth}{440pt}
\setlength{\topmargin}{0in}

\usepackage{amssymb}
\usepackage{amsfonts}
\usepackage{amsmath}

\title{Artificial Intelligence Note}
\author{Mengqi Zong $<mz2326@columbia.edu>$}

\begin{document}

\maketitle

\setlength{\parindent}{0in}

\section{Chapter 2 Intelligent agents}

1. An {\bf agent} is anything that can be viewed as perceiving environment throught sensors and acting upon that evnironment throught actuators. \\

2. Property of task environments

\begin{itemize}
  \item Fully observable vs. partially observable
  \item Signle angent vs. multiagent
  \item Deterministic vs. stochastic
  \item Episodic vs. sequential
  \item Static vs. dynamic
  \item Discrete vs. continuous
  \item Known vs. unknown
\end{itemize}

\section{Chapter 3 Solving Problems by Searching}

1. Uniformed search strategies

\begin{enumerate}
  \item Breadth-first search
  \item Uniform-cost search ($g(n)$)
  \item Depth-first search
  \item Depth-limited search
  \item Iterative deepening depth-first search
  \item Bidirectional search
\end{enumerate}

2. Infromed (heuristic) search strategies

\begin{enumerate}
  \item Greedy best-first search ($h(n)$)
  \item $A^*$ search \\
    Admissibility: An admissible heuristic is one that never overestimates the cost to reach the goal. \\
    Consistency (monotonicity): $h(n) \le c(n,a,n') + h(n')$. \\
    * Every consistent heuristic is also admissible.
  \item Iterative-deepening $A^*$
  \item Recursive best-first search (RBFS) \\
    backed-up value: the best $f$-value of its children
\end{enumerate}

\section{Chapter 4 Beyond Classical Search}

1. Local Search Algorithms \\

If the path to the goal does not matter... \\

Local search algorithms operate using a single current node (rather than multiple paths) and generally move only to neighbors of that node.

\begin{enumerate} 
  \item Hill-climbing (greedy local search) \\
    Stuck reasons: local maxima, ridges, plateaux. \\
    Alternatives: stochastic hill climbing, first-choice hill climbing, random restart hill climbing.
  \item Simulated annealing
  \item Local beam search \\
    In a local beam search, useful information is passed among the parallel search threads. \\
    Alternatives: stochastic beam search
  \item Genetic algorithms
\end{enumerate}

2. Searching with nondeterministic actions \\

AND-OR search trees \\

OR nodes: In a deterministic environment, the only branching is introduced by the agent's own choices in each state. \\
AND nodes: In a nondeterministic environment, branching is also introduced by the environment's choice of outcome for each action. \\

\section{Chapter 5 Adverserial Search}

1. Zero-sum (constant-sum) games: a game where the total payoff to all players is the same for every instance of the game. \\

2. The minimax algorithm \\

The minimax algorithm performs a complete depth-first exploration of the game tree. \\

3. Alpha-beta pruning

$\alpha =$ the value of the best (i.e., highest-value) choice we have found so far at any choice point along the path for MAX. \\
$\beta =$ the value of the best (i.e., lowest-value) choice we have found so far at any choice point along the path for MIN. \\

4. Imperfect real-time decisions: cut off search

\section{Chapter 7 Logical Agents}

1. A knowledge base is a set of sentences. Each sentence is expressed in a language called a knowledge representation language and represents some assertion about the world. A knowledge base, KB, may initially contain some background knowledge. \\

2. Entailment: a sentence follows logically from another sentence.

\begin{equation*}
  \alpha \models \beta \text{ if and only if } M(\alpha) \subseteq M(\beta)
\end{equation*}

if $\alpha \models \beta$, then $\alpha$ is a stronger assertion than $\beta$: it rules out more possible worlds. \\

Sound: an inference algorithm that derives only entailed sentences is called sound or truth-preserving. \\

Complete: an inference algorithm is complete if it can derive any sentence that is entailed. \\

3. Deduction theorem

\begin{equation*}
  \alpha \models \beta \text{ if and only if the sentence } (\alpha \Rightarrow \beta) \text{ is valid.}
\end{equation*}

\begin{equation*}
  \alpha \models \beta \text{ if and only if the sentence } (\alpha \lor \lnot \beta) \text{ is unsatisfiable.}
\end{equation*}

4. Modus Ponens

\begin{equation*}
  \frac {\alpha \Rightarrow \beta, \text{   } \alpha}{\beta}
\end{equation*}

5. And-Elimination

\begin{equation*}
  \frac {\alpha \land \beta}{\alpha} 
\end{equation*}

6. Unit resolution inference rule

\begin{equation*}
  \frac {\ell_1 \lor \cdots \lor \ell_k, \text{   } m}
        {\ell_1 \lor \cdots \lor \ell_{i-1} \lor \ell_{i+1} \lor \cdots \lor \ell_k}
\end{equation*}

where each $\ell$ is a literal and $\ell_i$ and $m$ are complementary literals. \\

7. Conjunctive normal form \\

(Disjunct) conjunct (Disjunct) \\

8. Ground resolution theorem \\

If a set of clauses is unsatisfiable, then the resolution closure of those clauses contains the empty clause. \\

9. Horn clause and definite clauses \\

Definite clauses: a disjunction of literals of which exactly one is positive. For example, the clause $(\lnot L_{1,1} \lor \lnot Breeze \lor B_{1,1})$. \\

Horn clause: a disjunction of literals of which at most one is positive. \\

Horn clauses are closed under resolution: if you resolve two Horn clauses, you get back a Horn clause. \\

\section{Chapter 8 First-order logic}

1. Terms: A term is a logical expression that refers to an object. For example,

\begin{equation*}
  LeftLeg(John)
\end{equation*}

A term with no variables is called a ground term. \\

2. Atomic sentences: an atomic sentence is formed from a predicate symbol optionally followed by a parenthesized list of terms, such as

\begin{equation*}
  Brother(Richard, John)
\end{equation*}

3. Complex sentences \\

We can use logical connectives to construct more complex sentences, with the same syntax and semantics as in proportional calculus. \\

4. Quantifiers \\

Uuniversial quantification ($\forall$) \\
Common mistake: use conjunction instead of implication with $\forall$ will lead to an overly strong statement.  \\

Existential quantification ($\exists$) \\
Common mistake: use implication instead of conjunction with $\exists$ will lead to a very weak statement. \\

5. Equality

\begin{equation*}
  Father(John) = Henry
\end{equation*}

6. Universal instantation (UI) \\

Every instantiation of a univerally quantified sentence is entailed by it:

\begin{equation*}
  \frac {\forall v \alpha}{Subst(\{v/g\}, \alpha)}
\end{equation*}

for any variable $v$ and ground term $g$. \\

7. Exisential instantation (EI) \\

For any sentence $\alpha$, variable $v$, and constant symbol $k$ that does not appear elsewhere in the knowledge base:

\begin{equation*}
  \frac {\exists v \alpha}{Subst(\{v/k\}, \alpha)}
\end{equation*}

E.g., $\exists x Crown(x) \land OnHead(x, John)$ yields:

\begin{equation*}
  Crown(C_1) \land OnHead(C_1, John)
\end{equation*}

provided $C_1$ is a new constant symbol, called a Skolem constant.

\section{Quantifying uncertainty}

1. Trying to use logic to cope with a domain like medical diagnosis fails for three main reasons: laziness, theoretical ignorance, practical ignorance. \\

2. Decision theory = probability theory + utility theory \\

3. Maximum expected utility (MEU): An agent is rational if and only if it chooses the action that yields the highest expected utility, averaged over all the possible outcomes of the action. \\

4. Sample space: the set of all possible worlds (possible worlds are mutally exclusive). \\

5.

\begin{equation}
  P(X|e) = \alpha P(X|e) = \alpha \sum_yP(X,e,y)
\end{equation}

\end{document}
