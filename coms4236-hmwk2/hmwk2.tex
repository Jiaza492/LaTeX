\documentclass[12pt]{article}

\parindent=.25in
\setlength{\oddsidemargin}{0pt}
\setlength{\textwidth}{440pt}
\setlength{\topmargin}{0in}

\usepackage{amsmath}
\usepackage[dvips]{graphicx}
\usepackage{verbatim}
\usepackage{appendix}

\title{COMS 4236 Homework 2}
\author{Mengqi Zong $<mz2326@columbia.edu>$}

\begin{document}

\maketitle

\setlength{\parindent}{0in}

\section*{Problem 1}

a.
\begin{itemize}
\item $L_1 \cup L_2$ \\
  Suppose $L_1$ can be recognized by nondeterministic Turing machine
  $M_1$, and $L_2$ can be recognized by nondeterministic Turing
  machine $M_2$. Then what we need to do here is simply simulating
  $M_1$ and $M_2$ at the same time by another NTM and then check their
  results.

  We only accept the input when $M_1$ and $M_2$ accept the input. If
  not, we reject the input. Note both simulations takes
  $NTIME(f(n))$, then the simulation takes $NTIME(f(n))$. And
  comparing the two results then generating the final result takes
  constant time, so the total time is still $NTIME(f(n))$.
\item $L_1 \cap L_2$ \\
  Similar to $L_1 \cap L_2$, the only difference is that we accept the
  input when one of $M_1$ and $M_2$ accept the input. If not, we
  reject the input. Obviously, the running time is $NTIME(f(n))$.
\item $L_1 \cdot L_2$ \\
  First, we guess the length of x. Note that if we know the length of
  x, then we will know the length of y.

  Second, try to recognize the first $length(x)$ input. If NTM
  accepts, then continue. If not, reject. If the second part of the
  input also get accepted by the NTM, then we accept the whole
  input. If not, we reject. In total, we just run the total time of
  $M_1$ and $M_2$ plus a constant. In sum, the total running time is
  $NTIME(f(n))$.
\end{itemize}

b. Basically, The Kleene star of a language $L$ can be treated as a
concatination of $L$: 

\begin{equation*}
\prod_{i=1}^m {x_i}, \; \; x_i \in L 
\end{equation*}

where m is the number of strings of language L in the input. \\

First, we guess how many strings of $L$ are there in the input. We
guess the length of the first string, then the second string, ... till
the total length reaches $n$. Second, we begin to recognize each of
the strings. If all of them are accepted by NTM, then we accept the
input. Note that the first part takes at most $NTIME(n)$, the second
part also takes $NTIME(n)$. As a result, the total running time of
$L^*$ is $NTIME(n)$, given that $L \in NTIME(n)$.

\section*{Question 2}

a. First, we can get that $TIME(2^n) \subseteq TIME(2^{2n})$ since
a TM of time complexity $2^{2n}$ can always simulate any TM of time
complexity $2^n$. Second, we can see that $2^n = o(2^{2n}) =
o((2^n)^2)$. Then from time hierarchy, we can know that there is a
language in $TIME(2^{2n}) - TIME(2^n)$. At last, we get $TIME(2^n)
\subset TIME(2^{2n})$. \\

b. Form the theorems that we learned on the relations between complete
classes, we know that $NTIME(f(n)) \subseteq SPACE(f(n))$. In our
case, we get $NTIME(n^2) \subseteq SPACE(n^2)$. \\

Similar to the proof of part a, we can get $SPACE(n^2) \subseteq
TIME(n^3)$. And by space hierarchy, we know that there is a language
in $SPACE(n^3) - SPACE(n^2)$. As last, we get $SPCE(n^2) \subset
SPACE(n^3)$, then $NTIME(n^2) \subset SPACE(n^3)$.

\section*{Question 3}

a. $\{n^k : k > 0 \}$ \\

$\{n^k : k > 0 \}$ are left closed.

\begin{eqnarray*}
  p(n^k) &=& \Theta {((n^k)^c)} \\
  &=& \Theta {(n^{ck})}
\end{eqnarray*}

where $c$ is a constant. In this case, for every constant $c$, we can
always find
\begin{eqnarray*}
  g(n) &=& n^{ck}
\end{eqnarray*}

such that

\begin{eqnarray*}
  p(n^k) &=& \Theta {(n^k)^c} \\
  &\ge& c' \cdot n^{ck} \\
  &=& O(g(n))
\end{eqnarray*}

where $c'$ is a constant. As we can see, $\{ n^k : k > 0 \}$ are
closed under left composition.

b. $\{ k \cdot n : k > 0 \}$ \\
$\{ k \cdot n : k > 0 \}$ are left closed.
\begin{eqnarray*}
  p(k \cdot n) &=& \Theta {((k \cdot n)^c)} \\
  &=& \Theta {(n^c)}
\end{eqnarray*}

where $c$ is a constant. In this case, for every constant $c$, we can
always find

\begin{eqnarray*}
  g(n) &=& n^c
\end{eqnarray*}

such that 

\begin{eqnarray*}
  p(k \cdot n) &=& \Theta {(n^c)} \\
  &\ge& c' \cdot n^c \\
  &=& O(g(n))
\end{eqnarray*}

where $c'$ is a constant. As we can see, $\{ k \cdot n : k > 0 \}$ are
closed under left composition. \\

f. $\{\log {n} \}$ \\

$\{\log {n} \}$ are not left closed. For example, you can never find a
$g(n) \in \{\log {n} \}$ such that $(\log {n})^{0.2} = O(g(n))$. \\

$\{\log {n} \}$ are right closed. We have

\begin{eqnarray*}
  \log {p(n)} &=& \log {\Theta {(n^c)}} \\
  &=& c \log {\Theta {(n)}}
\end{eqnarray*}

where $c$ is a constant. In this case, let $g(n) = \log {n}$. Then for
any constant c, we have

\begin{eqnarray*}
  \log {p(n)} &=& \log {\Theta {(n^c)}} \\
  &=& c \log {\Theta {(n)}} \\
  &\ge& c \log {c_1 \cdot n}\\
  &=& O(g(n))
\end{eqnarray*}

where $c_1$ is a constant. As we can see, $\{\log {n} \}$ are
closed under right composition. \\

\section*{Question 4}






\end{document}
