\documentclass[12pt]{article}

\parindent=.25in
\setlength{\oddsidemargin}{0pt}
\setlength{\textwidth}{440pt}
\setlength{\topmargin}{0in}

\usepackage{amsmath}
\usepackage{amsfonts}
\usepackage[dvips]{graphicx}
\usepackage{verbatim}
\usepackage{appendix}

\title{COMS 6253 Problem 1}
\author{Mengqi Zong $<mz2326@columbia.edu>$}

\begin{document}

\maketitle

\setlength{\parindent}{0in}

{\bf Problem 1.} The parity function on k 0/1-valued variables is

\begin{equation*}
PAR(x_1,...,x_k) = x_1 + ... + x_k \text { mod } 2,
\end{equation*}


a)  Show that the parity function on $\log s$ variables can be
computed by a decision tree of size $s$. \\

Since decision trees are a universal representation scheme, any
Boolean function with $k$ variables can be computed by a complete
decision tree of depth $k$ with $2^k$ leaves that exhaustively queries
all $k$ variables on every path. \\

For the parity function with $\log s$ variables, its complete decision
tree has depth $\log s$ with $2^{\log s} = s$ leaves. That is, the
parity function on $\log s$ variables can be computed by a decision
tree of size $s$. \\

b)  Show that any PTF for the parity function on $k$ variables must
have degree at least k. \\



\end{document} 
