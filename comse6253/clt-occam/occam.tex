\documentclass[12pt]{article}
\parindent=.25in

\setlength{\oddsidemargin}{0pt}
\setlength{\textwidth}{440pt}
\setlength{\topmargin}{0in}
\usepackage{amssymb}
\usepackage{amsfonts}
\usepackage{amsmath}
\usepackage{latexsym}
\usepackage[center]{subfigure}
\usepackage{epsfig}
\usepackage{hyperref}

\usepackage{amsthm}

\newtheorem{theorem}{Theorem}
\newtheorem{lemma}[theorem]{Lemma}
\newtheorem{remark}{Remark}
\newtheorem{fact}[theorem]{Fact}
\newtheorem{definition}[theorem]{Definition}
\newtheorem{corollary}[theorem]{Corollary}
\newtheorem{proposition}[theorem]{Proposition}
\newtheorem{claim}[theorem]{Claim}
\newtheorem{conjecture}[theorem]{Conjecture}
\newtheorem{observation}[theorem]{Observation}
\newtheorem{assumption}[theorem]{Assumption}
\newtheorem{example}[theorem]{Example}

\newcommand{\noi}{{\noindent}}
\newcommand{\ms}{{\medskip}}
\newcommand{\msni}{{\medskip \noindent}}

\newcommand{\la}{\langle}
\newcommand{\ra}{\rangle}
\newcommand{\calc}{{\cal C}}
\newcommand{\cald}{{\cal D}}
\newcommand{\calh}{{\cal H}}
\newcommand{\cala}{{\cal A}}

\newcommand{\sign}{\mathrm{sign}}
\newcommand{\eps}{\epsilon} 
\newcommand{\poly}{\mathrm{poly}}
\newcommand{\size}{\mathrm{size}}
\newcommand{\depth}{\mathrm{depth}} 

\pagestyle{headings}    % Go for customized headings

\newcommand{\handout}[5]{
   \noindent
   \begin{center}
   \framebox{
      \vbox{
    \parbox[t]{4in} {\bf #1 } \vspace{3mm}  {\hfill \bf #2 }
       \vspace{2mm}
       \hbox to 6.00in { {\Large \hfill #5  \hfill} }
       \vspace{1mm}
       \hbox to 6.00in { {\it #3 \hfill #4} }
      }
   }
   \end{center}
   \vspace*{1mm}
}

\begin{document}

\setlength{\parindent}{0in}

\handout{COMS 6253: Advanced Computational Learning Theory}{Spring 2012}
{Lecturer: Rocco Servedio}
{Scribe: Mengqi Zong}{PAC-learning, Occam Algortihm and Compression}

\thispagestyle{plain}

\section{Introduction}

The distribution-independent model of concept learning (PAC-learning,
for ``probably approximately correct learning'') was introduced by
Valiant and has been widely used to investigate the phenomenon of
learning from examples. \\

In COMS 6253, we spent most of our time studying PAC-learning
model. It's worthy for us to spend some time to take a closer look at
this model. \\

This lecture will talk about the relationship between PAC-learning
model and Occam algorithm. Later, we will show that for many classes,
PAC-learnability is equivalent to data compression.

\section{PAC-learning and Occam algorithm}

\subsection{Notation and Definitions}

We describe a context for representing concepts over $X$. We define a
class of representations to be a four-tuple ${\bf R} = (R, \Gamma, c,
\Sigma)$. $\Sigma$ and $\Gamma$ are sets of characters. Strings
composed of characters in $\Sigma$ are used to describe elements of
$X$, and strings of characters in $\Gamma$ are used to describe
concepts. $R \subseteq \Gamma^*$ is the set of strings that are
concept descriptions or representations into concepts over
$\Sigma$. $R$ may be thought of as a collection of names of concepts,
and for any $r \in R$, $c(r) \subseteq \Sigma^*$ is the concepts named
by $r$. \\

We let $R^{[s]}$ denote the set $\{ r \in R: |r| \le s\}$; that is,
the set of all representations from R of length at most $s$. If ${\bf
  R} = (R, \Gamma, c, \Sigma)$ is a class of representations, $r \in
R$, and $D$ is a probability distribution on $\Sigma^*$, then
$EXAMPLE(D, r)$ is an oracle that, when called, randomly chooses an $x
\in \Sigma^*$ according to distribution $D$ and returns the pair $(x,
r(x))$. \\

A randomized algorithm is an algorithm that behaves like a
deterministic one with the additional property that, at one or more
steps during its execution, the algorithm can flip a fair two-sided
coin and use the result o the coin flip in its ensuing computation.

\begin{definition}
The representation class ${\bf R} = (R, \Gamma, c, \Sigma)$ is
PAC-learnable if there exists a (possibly randomized) algorithm L and
a polynomial $p_L$ such that for all $s, n \ge 1$, for all $\epsilon$
and $\delta$, $0 < \epsilon, \delta < 1$, for all $r \in R^{[s]}$, and
for all probability distributions $\cald$ on $\Sigma^{[n]}$, if L is
given as input the parameters $s, \epsilon$, and $\delta$, and may
access the oracle EXAMPLE($\cald$, r), then L halts in time $p_L(n, s,
1/\epsilon, 1/\delta)$ and, with probability at least $1 - \delta$,
outputs a representation $r' \in R$ such that $\cald (r' \oplus r) \le 
\epsilon$. Such an algorithm L is a polynomial-time learning algorithm
for {\bf R}.
\end{definition}

\begin{definition}
A randomized polynomial-time (length-based) Occam algorithm for a
class of representations ${\bf R} = (R, \Gamma, c, \Sigma)$ is a
(possibly randomized) algorithm $O$ such that there exists a constant
$\alpha < 1$ and a polynomial $p_O$, and such that for all $m, n, s
\ge 1$ and $r \in R^{[s]}$, if $O$ is given as input any sample $M
\subseteq S_{m,n,r}$ and $1 / \gamma$, and, with probability at least
$1 - \gamma$, outputs a representation $r' \in R$ that is consistent
with M, and such that $|r'| \le p_O(n, s, 1 / \gamma)m^{\alpha}$.
\end{definition}

\subsection{Example: Learning Boolean conjunctions}

In the book ``An Introduction to Computational Learning Theory'',
learning Boolean conjunctions has been given as an example to compare
PAC-learning and Occam algorithm. \\

For PAC-learning, the regular approach is to use inequality and union
bounds to calculate a probability of $m$ examples satisfy the
needs. And the number of examples from PAC-learning is

\begin{eqnarray*}
m_{PAC} \ge (2n / \epsilon) (\ln {(2n)} + \ln {(1 / \delta)})
\end{eqnarray*}

As for Occam's algorithm, the approach is simply to take a look at the
representation class $H_n$ of Boolean conjunctions and see what we can
get there. The number of examples from Occam' algorithm is

\begin{eqnarray*}
m_{Occam} \ge \frac {1}{\epsilon} \log {\frac {1}{\delta}} + n /
\epsilon
\end{eqnarray*}

Note that Occam's algorithm is an improvement by a logarithmic factor
over the bound $m_{PAC}$. But the improvement doesn't happen
always. For this example, the reason why Occam's algorithm gives a
tighter bound is probably due to the union bounds used in the
PAC-learning. \\

As we can see, Occam's algorithm is a naive approach for learning. And
on most cases, the bound given by Occam's algorithm is a general
complexity bound for PAC-learning.

\subsection{The sufficient condition}

We now show that the existence of an Occam algorithm for a class of
concepts is a sufficient condition for the PAC-learnability of that
class. 

\begin{theorem}
Let ${\bf R} = (R, \Gamma, c, \Sigma)$ be a class of representations
with $\gamma$ finite. If there exists a randomized polynomial-time
(length-based) Occam algorithm for ${\bf R}$, then ${\bf R}$ is
PAC-learnable.
\end{theorem}

This theorem shows that the existence of an Occam algorithm for a
class of concepts is a sufficient condition for the PAC-learnability
of that class. \\

However, it was left as an {\bf open problem} whether PAC-learnability
is equivalent to the existence of Occam algorithms; i.e. whether the
existence of an Occam algorithm is also a necessary condition for 
PAC-learnability.

\subsection{Exception list}

\begin{definition}
A class ${\bf R} = (R, \Gamma, c, \Sigma)$ is polynomially closed
under exception lists if there exists an algorithm EXLIST and a
polynomial $p_{EX}$ such that for all $n \ge 1$, on input of any $r
\in R$ and any finite set $E \subset \Sigma^{[n]}$, EXLIST halts in
time $p_{EX}(n, |r|, |E|)$ and outputs a representation $EXLIST(r, E)
= r_E \in R$ such that $c(r_E) = c(r) \oplus E$. Note that the
polynomial running time of EXLIST implies that $|r_E| \le p_1(n, |r|,
\log |E|) + p_2(n, \log |r|, \log |E|)|E|$ is satisfied, then we say
that ${\bf R}$ is strongly polynomially closed under exception lists.
\end{definition}

Clearly, any representation class that is strongly polynomially closed
is also polynomially closed. The definition of polynomial closure is
above is easily understood - it asserts that the representation $r_E$
that incorporates exceptions $E$ into the representation r has size at
most polynomially larger than the size of $r$ and the total size of
$E$, the latter of which is at most $n|E|$.

\subsection{Results for finite representation alphabets}

We consider the case in which the alphabet $\Gamma$ (over which the
representations of concepts are described) is finite. This typically
occurs when concepts are defined over discrete domains (e.g. Boolean
formulas, automata, etc.).

\begin{theorem}
If ${\bf R} = (R, \Gamma, c, \Sigma)$ is strongly polynomially closed
under exception lists and ${\bf R}$ is PAC-learnable, then there
exists a randomized polynomial-time (length-based) Occam algorithm for
${\bf R}$.
\end{theorem}

\begin{corollary}
Let $\Gamma$ be a finite alphabet. If ${\bf R} = (R, \Gamma, c,
\Sigma)$ is strongly polynomially closed under exception lists, then
${\bf R}$ is PAC-learnable if and only if there exists a randomized
polynomial-time (length-based) Occam algorithm for ${\bf R}$.
\end{corollary}

\section{PAC-learning and Compression}

\subsection{Learning versus Prediction}

There are two common variants on the standard definition of
PAC-learning. Under these alternative definitions, the hypothesis
output by the learning is not required to be of the same form as the
target concept description. \\

The notion of learning one representation class ${\bf R} = (R, \Gamma,
c, \Sigma)$ in terms of another representation class ${\bf R'} = (R',
\Gamma', c', \Sigma')$ was introduced. Under this definition, a
learning algorithm for ${\bf R}$ is required to output hypotheses in
$R'$ rather than in R. A representation class ${\bf R}$ is
polynomially predictable if there exists a representation class R'
with a uniform polynomial-time evaluation procedure procedures such
that ${\bf R}$ is PAC-learnable.

\begin{example}
Fixed encoding and Huffman encoding.
\end{example}

\subsection{Data compression}

Now we talk about the relationship between learning and data
compression. It has been shown that, if any sample can be
compressed-that is, represented by a prediction rule significantly
smaller than the original sample-then this compression algorithm can be
converted into a PAC-learning algorithm. \\

Suppose $C_n$ is a learnable concept class and that we have been given
$m$ examples $(v_1, c(v_1)), (v_2, c(v_2)),..., (v_m, c(v_m))$ where
each $v_i \in X_n$ and $c$ is a concept in $C_n$ of size $s$. These
examples need not have been chosen at random. The data compression
problem is to find a small representation for the data, that is, an
hypothesis $h$ that is significantly smaller than the original data
set with the property that $h(v_i) = c(v_i)$for each $v_i$. An
hypothesis with this last property is said to be consistent with the
sample.

\begin{theorem}
Let C be a learnable concept class. Then there exists an efficient
algorithm that, given $0 < \delta \le 1$ and $m$ (distinct) examples
of a concept $c \in C_n$ of size $s$, outputs with probability at
least $1 - \delta$ a deterministic hypothesis consistent with the
sample and of size polynomial in $n$, $s$, and $\log m$.
\end{theorem}

\subsection{Future Research}

\begin{itemize}
\item Motivation: it is useful.

  Machine learning techniques have been widely used in lots of fields,
  including data compression. And it seems quite hard to do the
  analysis from a theoretical perspective. We need a theoretical model
  to help us do the analysis.
\item PAC-learning and lossy data compression: graphics.

  The symmetric difference of $c$ and $h$: $c \Delta h = (c \backslash
  h) \cup (h \backslash c)$. We could drop the part we don't care-the
  $\epsilon$ part.
\item Exact-learning and lossless data compression: r-junta.

  The general description length of an r-junta
  \begin{eqnarray*}
    DL_{general} = n \cdot \log n + 2^n
  \end{eqnarray*}
  The probably best description length of an r-junta
  \begin{eqnarray*}
    DL_{best} = r \cdot \log r + 2^r
  \end{eqnarray*}
  As we can see, by using $DL_{best}$, we achieve a compression rate of
  \begin{eqnarray*}
    r \approx = 1 - \frac {2^r}{2^n} = 1 - 2^{r-n}
  \end{eqnarray*}
  That is, we saved about $1 - 2^{r-n} \time 100\%$ of the space.
\item The compression model is already complicated enough.

  This is a long and winding road.
\end{itemize}

\section{References}

\begin{itemize}
\item Raymond Board, Leonard Pitt (1990). \emph {On the necessity of
    Occam algorithms}.
\item Robert E. Schapire (1990). \emph {The Strength of Weak
    Learnability}.
\item Micheal J. Kearns, Umesh V. Vazirani. \emph {An introduction to
    computational learning theory}.
\item Elchanan Mossel, Ryan O'Donnell, Rocco A. Servedio (2003). \emph
  {Learning Juntas}.
\end{itemize}

\end{document}
