\documentclass[12pt]{article}
\parindent=.25in

\setlength{\oddsidemargin}{0pt}
\setlength{\textwidth}{440pt}
\setlength{\topmargin}{0in}

\usepackage{amsmath}
\usepackage[dvips]{graphicx}
\usepackage{verbatim}
\usepackage{appendix}

\usepackage{amssymb}
\usepackage{amsfonts}
\usepackage{latexsym}
\usepackage[center]{subfigure}
\usepackage{epsfig}
\usepackage{hyperref}

\title{Stat 4201 Homework 6}
\author{Mengqi Zong $<mz2326@columbia.edu>$}

\begin{document}

\maketitle

\setlength{\parindent}{0in}

\section*{Question 1}

Due to different sample sizes, we will use pooled estimate of standard
deviation:

\begin{eqnarray*}
SD
&=& \sigma \sqrt {\frac {1}{n_1} + \frac{1}{n_2}} \\
&=& \sigma \sqrt {\frac {1}{n_1} + \frac{1}{rn_1}} \\
&=& \frac {\sigma}{\sqrt {n_1}} \sqrt {1 + 1 / r} 
\end{eqnarray*}

From
\begin{equation*}
Z_{\alpha} + \frac {\triangle}{\text {SD}} = - Z_{\beta}
\end{equation*}

We get

\begin{eqnarray*}
Z_{\alpha} + \frac {\triangle}{\frac {\sigma}{\sqrt {n_1}} \sqrt {1 +
    1 / r}} = - Z_{\beta} \\
Z_{\alpha} + \frac {\sqrt {n_1} \triangle}{\sigma \sqrt {1 +
    1 / r}} = - Z_{\beta} \\
\frac {\sqrt {n_1} \triangle}{\sigma \sqrt {1 + 1 / r}} = -
(Z_{\alpha} + Z_{\beta}) \\
\sqrt {n_1} = - \frac {\sigma \sqrt {1 + 1 / r}(Z_{\alpha} +
  Z_{\beta})}{\triangle} \\
n_1= \frac {\sigma^2 (1 + 1 / r) (Z_{\alpha} +
  Z_{\beta})^2}{\triangle^2}
\end{eqnarray*}

\section*{Question 3}

a) Here is the function to compute the sample size for t-test:

\begin{verbatim}
ssize.t.test<-function(sig.level = 0.05, power = 0.9, delta = 5,
                       sigma = 3, alt = "two.sided") 
{
  # check the alternative
  if (alt == "two.sided") {
    alpha = sig.level / 2 
  } else if (alt == "one.sided"){
    alpha = sig.level
  } else {
    print("Warning: incorrect alt")
    return(NULL)
  }

  beta = 1 - power
  
  size= 2 * ((qnorm(alpha) + qnorm(beta))^2 * sigma^2) / delta^2

  return(ceiling(size))  
}
\end{verbatim}

Using the function, we get

\begin{verbatim}
> ssize.t.test(sig.level = 0.05, power = 0.9, delta = 5, sigma = 3,
alt = "two.sided")
[1] 8
\end{verbatim}

As we can see, the sample size for each group is 8. \\

We can also compute the sample size using R function ``power.t.test'',
we get

\begin{verbatim}
> power.t.test(power = 0.9, delta = 5, sd = 3)

     Two-sample t test power calculation 

              n = 8.649245
          delta = 5
             sd = 3
      sig.level = 0.05
          power = 0.9
    alternative = two.sided

 NOTE: n is number in *each* group 
\end{verbatim}

As we can see, the sample size for each group is 9. \\

At last, we draw the conclusion that the sample size is 9.\\

b) Here is the function to compute the sample size for proportion
test:

\begin{verbatim}

ssize.prop.test<-function(p1 = 0.6, p2 = 0.75, sig.level = 0.05,
                          power = 0.9, alt = "two.sided") 
{
  # check the alternative
  if (alt == "two.sided") {
    alpha = sig.level / 2 
  } else if (alt == "one.sided"){
    alpha = sig.level
  } else {
    print("Warning: incorrect alt")
    return(NULL)
  }

  beta = 1 - power
  p = (p1 + p2) / 2
  q = 1 - p
  q1 = 1 - p1
  q2 = 1 - p2
  delta = p2 - p1

  molecular = (qnorm(alpha) * sqrt(2 * p * q)
               + qnorm(beta) * sqrt(p1 * q1 + p2 * q2))^2 
  
  size= molecular / delta^2

  return(ceiling(size))  
}
\end{verbatim}

Using this function, we get

\begin{verbatim}
> ssize.prop.test(p1 = 0.6, p2 = 0.75, sig.level = 0.05, power = 0.9,
alt = "two.sided")
[1] 203
\end{verbatim}

As we can see, the sample size for each group is 203. \\

We can also use the R function ``power.prop.test'' to calculate the
sample size:

\begin{verbatim}
> power.prop.test(p1 = 0.6, p2 = 0.75, sig.level = 0.05, power = 0.9)

     Two-sample comparison of proportions power calculation 

              n = 202.8095
             p1 = 0.6
             p2 = 0.75
      sig.level = 0.05
          power = 0.9
    alternative = two.sided

 NOTE: n is number in *each* group 
\end{verbatim}

As we can see, the sample size for each group is 203. \\

At last, the sample size for each group is 203. \\

c) Using the function in part b, we get

\begin{verbatim}
> ssize.prop.test(p1 = 1/4000, p2 = 1/12000, sig.level = 0.05, power =
0.9, alt = "two.sided")
[1] 126066
\end{verbatim}

As we can see, the sample size for each group is 126066. \\

We can also use the R function ``power.prop.test'' to calculate the
sample size:

\begin{verbatim}
> power.prop.test(p1 = 1/4000, p2 = 1/12000, sig.level = 0.05, power =
0.9)

     Two-sample comparison of proportions power calculation 

              n = 126066
             p1 = 0.00025
             p2 = 8.333333e-05
      sig.level = 0.05
          power = 0.9
    alternative = two.sided

 NOTE: n is number in *each* group 
\end{verbatim}

As we can see, the sample size for each group is 126066. \\

At last, the sample size for each group is 126066.

\end{document} 
