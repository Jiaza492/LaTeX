\documentclass[12pt]{article}

\parindent=.25in
\setlength{\oddsidemargin}{0pt}
\setlength{\textwidth}{440pt}
\setlength{\topmargin}{0in}

\usepackage{amsmath}
\usepackage[dvips]{graphicx}
\usepackage{verbatim}
\usepackage{appendix}

\title{COMS 4236 Homework 4}
\author{Mengqi Zong $<mz2326@columbia.edu>$}

\begin{document}

\maketitle

\setlength{\parindent}{0in}

\section*{Problem 1}


\section*{Problem 2}

a. The kernel can't contain none of $u, v$. \\

If the kernel $K$ does not include any nodes in $u$ and $v$, then condition (2) of a kernel can't hold: $u,v \notin K$ and there are no other edges entering the nodes $u, v$. For nodes $u, v$, they are neither in $K$ nor have an incoming edge from some node in K. \\

The kernel cant' contain both $u$ and $v$. \\

If the kernel $K$ contains both $u$ and $v$, then condition (1) of a kernel can't hold: for nodes $u, v$, $u,v \in K$ and there are edge (u,v). \\

To sum up, any kernel of G must include exactly one of the two nodes $u,v$. \\

b. Suppose there exists a kernel $K$ of G such that any node in $K$ does not contain some node (distinct from $u, v, w$) that has an edge to at least one of the three nodes $u, v, w$. \\

In this case, none of $u, v, w$ is in $K$. Because nodes $u, v, w$ form a cycle $u \to v \to w \to u$, if any node of $u, v, w$ is in $K$, then there would be at least one edge to all the three nodes, which violates the condition (1) of a kernel. \\

If there is no such edge that comes from some node in $K$ that is distinct from the three nodes, then the condition (2) of a kernel can't hold for kernel $K$: $u, v, w \notin K$ and $u, v, w$ also have no incoming edge from some node in $K$. 

So, the assumption is not correct. And any kernel of G must contain some node x (distinct from $u, v, w$) that has an edge to at least one of the three nodes $u, v, w$. \\

c. We can reduce 3SAT to this problem. \\

For a given 3SAT problem, without loss of generality, suppose there are n variables $x_1, x_2, ..., x_n$ and m disjunction terms $t_1, t_2, ..., t_m$.

\begin{itemize}
\item First, we create $2n$ nodes to represent all the literals: $n_{x_1}, n_{\bar {x_1}}, ..., n_{x_n}, n_{\bar {x_n}}$. And for all $i = 1, 2, ..., n$, we add edges $(n_{x_i}, n_{\bar{x_i}})$ and $(n_{\bar{x_i}}, n_{x_i})$ to the graph.
\item Second, for all $m$ disjunction terms, we create $m$ nodes to represent each term $n_{t_1}, n_{t_2}, ..., n_{t_m}$. And for every term $t_j, j = 1, 2, ..., m$, we do the following: suppose $t_j = x_a \lor x_b \lor \bar {x_c}$, then we add three edges $(n_{x_a}, n_{t_j}), (n_{x_b}, n_{t_j}), (n_{\bar {x_c}}, n_{t_j})$ to the graph.
\item Third, we draw a cycle $n_{t_1} \to n_{t_2} \to ... \to n_{t_m} \to n_{t_1}$ in the graph.
\end{itemize}

Then we can prove that the kernel of the new graph consists of $n$ nodes which are representing the literals of the 3SAT, and the $n$ literals is the solution of the given 3SAT problem.

\begin{itemize}
\item For every pair of nodes $n_{x_i}$ and $n_{\bar {x_i}}$, we have edges $(n_{x_i}, n_{\bar{x_i}})$, $(n_{\bar{x_i}}, n_{x_i})$ and there are no other edges entering the nodes $n_{x_i}$ and $n_{\bar {x_i}}$. From the result of part a, we know that the kernel of the new graph must contain exactly one of the two nodes.
\item From the result of part b, we can easily prove the similar result that applies to a cycle with m nodes, like $n_{t_1} \to n_{t_2} \to ... \to n_{t_m} \to n_{t_1}$. In this case, we know that all the nodes representing the terms can't be in the kernel of the graph. Also, 
\end{itemize}





\section*{Problem 3}

a.


\section*{Problem 4}

a. In order to satisfy a DNF, we just need to satisfy one conjunction term. We can choose the first conjunction term of the DNF, if $x_i$ is in this term, set $x_i$ to be 1; if $\bar {x_i}$ is in this term, set $x_i$ to be 0. As to the rest, we can arbitrarily set the value, for example, all set to 0. \\

In order to get the result, we at most need to scan entire the input (when there is only one conjunction term in the DNF). So, the Satisfiability problem for Boolean formulas in DNF is in P. \\

b. \\

\section*{Problem 5}




\end{document}
