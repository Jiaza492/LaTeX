\documentclass[12pt]{article}

\parindent=.25in
\setlength{\oddsidemargin}{0pt}
\setlength{\textwidth}{440pt}
\setlength{\topmargin}{0in}

\usepackage{amsmath}
\usepackage[dvips]{graphicx}
\usepackage{verbatim}
\usepackage{appendix}

\title{COMS 4236 Homework 4}
\author{Mengqi Zong $<mz2326@columbia.edu>$}

\begin{document}

\maketitle

\setlength{\parindent}{0in}

\section*{Problem 1}

First, Partition Problem is in NP. We can easily verify that, for a given
partition, if the two sums are equal in poly-time. \\

Second, Partition Problem is NP-hard. We can reduce Subset Sum to
Partition Problem. \\

Let the sum of all integers in S be $s$. If $t \ge s$, then we solved
this problem. If $t < s$, then

\begin{itemize}
\item If $t = \frac {s}{2}$, then the Subset Sum is a Partition
  Problem.
\item If $t < \frac {s}{2}$, then we create a new $P = S \cup \{ s-2t
  \}$. And we try to see if P can be equally partitioned. Since the
  sum of the integers in P is $2s-2t$, then if P can be equally
  partitioned, then the sum of each subcollection is $s -
  t$. Since number $s-2t$ is in one of the subcollections, then the
  set S must have a subcollection whose sum is t.
\item If $t > \frac {s}{2}$, then we create a new $P = S \cup \{ 2t-s
  \}$. And we try to see if P can be equally partitioned. Since the
  sum of the integers in P is $2t-2s$, then if P can be equally
  partitioned, then the sum of each subcollection is $t$. Since number
  $s-2t$ can only be in one of the subcollections, then the set S must
  have a subcollection whose sum is t.
\end{itemize}

As we can see, the reduction takes polynomial time.

To sum up, Partition Problem is NP-complete. \\

\section*{Problem 2}

a. The kernel can't contain none of $u, v$. \\

If the kernel $K$ does not include any nodes in $u$ and $v$, then
condition (2) of a kernel can't hold. Because $u,v \notin K$ and there are no
other edges entering the nodes $u, v$. For nodes $u, v$, they are
neither in $K$ nor have an incoming edge from some node in K. \\

The kernel cant' contain both $u$ and $v$. \\

If the kernel $K$ contains both $u$ and $v$, then condition (1) of a
kernel can't hold: for nodes $u, v$, $u,v \in K$ and there are edge
(u,v). \\ 

To sum up, any kernel of G must include exactly one of the two nodes
$u,v$. \\

b. Suppose there exists a kernel $K$ of G such that any node in $K$
does not contain some node (distinct from $u, v, w$) that has an edge
to at least one of the three nodes $u, v, w$. \\ 

In this case, none of $u, v, w$ is in $K$. Because nodes $u, v, w$
form a cycle $u \to v \to w \to u$, if any node of $u, v, w$ is in
$K$, then there would be at least one edge to all the three nodes,
which violates the condition (1) of a kernel. \\ 

If there is no such edge that comes from some node in $K$ that is
distinct from the three nodes, then the condition (2) of a kernel
can't hold for kernel $K$: $u, v, w \notin K$ and $u, v, w$ also have
no incoming edge from some node in $K$.  

So, the assumption is not correct. And any kernel of G must contain
some node x (distinct from $u, v, w$) that has an edge to at least one
of the three nodes $u, v, w$. \\ 

c. We can reduce 3SAT to this problem. \\

For a given 3SAT problem, without loss of generality, suppose there
are n variables $x_1, x_2, ..., x_n$ and m disjunction terms $t_1,
t_2, ..., t_m$. 

\begin{itemize}
\item First, we create $2n$ nodes to represent all the literals:
  $n_{x_1}, n_{\bar {x_1}}, ..., n_{x_n}, n_{\bar {x_n}}$. And for all
  $i = 1, 2, ..., n$, we add edges $(n_{x_i}, n_{\bar{x_i}})$ and
  $(n_{\bar{x_i}}, n_{x_i})$ to the graph. 
\item Second, for each disjunction term, we create three new nodes $u,
  v, w$ to form a cycle $u \to v \to w$. The three new nodes represent
  the three literals forming the term. And then we create three edges
  from the literals to the repsective nodes in this new circle. For
  example, $u, v, w$ represent literal $x_1, \bar {x_2}, x_3$, then
  we add edges $(n_{x_1}, u), (n_{\bar {x_2}}, v), (n_{x_3}, w)$ to
  the graph.
\end{itemize}

Then we can prove that the kernel of the new graph contains exactly $n$
nodes from $n_{x_1}, n_{\bar {x_1}}, ..., n_{x_n}, n_{\bar {x_n}}$,
and the $n$ literals is the solution of the given 3SAT problem.

\begin{itemize}
\item For every pair of nodes $n_{x_i}$ and $n_{\bar {x_i}}$, we have
  edges $(n_{x_i}, n_{\bar{x_i}})$, $(n_{\bar{x_i}}, n_{x_i})$ and
  there are no other edges entering the nodes $n_{x_i}$ and $n_{\bar
    {x_i}}$. From the result of part a, we know that the kernel of the
  new graph must contain exactly one of the two nodes.
\item As to the circles representing the disjunction terms, from the
  part b, we know that the kernel must contain some node x (distinct
  from the nodes forming that circle) that has an edge to at least one
  of the three nodes. As we can see, the ``some'' node x to this
  disjunction term is the value to satisfy this term.
\end{itemize}

\section*{Problem 3}

a. Suppose the optimal graph has cycles. Assume there's a circle $u
\to v \to ... \to w \to u$ in the graph, then we can remove one edge
in the circle and still get all nodes connected. Since all distances
are positive integers, the modified graph has less total distance
than the previous optimal graph. In this case, the optimal graph with
cycles is not optimal, which contradicts with the assumption. So, the
optimal graph is a tree (every node is connected). \\

b. Given a number $a$, set $N = \{1,...,n\}$ of n cities, a subset $M
\subseteq N$ of mandatory cities, and the pairwise distances
$d(i,j)>0, 1 \le i,j \le n$ between the cities, which are assumed to
be positive integers and symmetric. The problem is to find out if
there is a connected graph H=(V,E) that includes all mandatory cities
and which has total distance $d(H) = \Sum \{d(i,j)|(i,j) \ in
E \}$ that is at most $a$? \\

c. Basically, we will use binary search to do solve this optimization
problem. \\

Given input with size n, we can conclude that the maximum total
distance will have no more than $n$ bits, that is, no more than
$2^n$. Because the input contains all distances and bits represent
edges, nodes. And the sum of the distances can't be a number that
costs more than $n$ bits. \\

Then what we do is to first calculate the total distance $d$, then use
the subroutine to check if $\frac {d}{2}$. If $\frac {d}{2}$ returns
1, then check $frac {\frac {d}{2} + d}{2}$; if $\frac {d}{2}$ returns 0,
then check $\frac {\frac {d}{2} + 0}{2}$. And so on, until we find the
optimal solution. This procedure takes $O(\log {(n)})$ times, and
every time we call the subroutine which takes polynomial time. So in
total, this algorithm takes polynomial time. \\

d. 


\section*{Problem 4}

a. In order to satisfy a DNF, we just need to satisfy one conjunction
term. We can choose the first conjunction term of the DNF, if $x_i$ is
in this term, set $x_i$ to be 1; if $\bar {x_i}$ is in this term, set
$x_i$ to be 0. As to the rest, we can arbitrarily set the value, for
example, all set to 0. \\

In order to get the result, we at most need to scan entire the input
(when there is only one conjunction term in the DNF). So, the
Satisfiability problem for Boolean formulas in DNF is in P. \\

b. The complementary problem of DNF-TAUT is $CP = \{ \text{DNF formula
} F \; | \; \exists \text { assignment } x, F(x) = 0 \}$. We will
prove that DNF-TAUT is coNP-complete by proving CP is NP-complete. \\ 


First, CP is in NP, because we can verify its result in polynomial
time. \\

Also, We can reduce SAT to CP. For any given SAT problem, finding a
satisfiable assignment is equivalent to finding an unsatisfiable
assignment of its negation. And we can transform the negation of a CNF
into a DNF in polynomial time. All we need to do is to change all
$\land$ into $\lor$, change $lor$ into $land$ and replace literal $x$
with its counterpart $\bar {x}$.

To sum up, CP is NP-complete. Since DNF-TAUT is the complementary
problem of CP, then DNF-TAUT is coNP-complete.

\section*{Problem 5}




\end{document}
